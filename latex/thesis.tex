\documentclass[pdfa,cucitura]{toptesi}

\hypersetup{
    pdfpagemode={UseOutlines},
    bookmarksopen,
    pdfstartview={FitH},
    colorlinks,
    linkcolor={blue},
    citecolor={red},
    urlcolor={blue}
}

%
% Phantom space for abbreviations
%
\usepackage{xspace}
%
% To insert doi identifiers
%
\usepackage{doi}
\renewcommand{\doitext}{DOI }
%
% Improve citations from biblio
%
\usepackage{cite}
%
% This is to create hyperlinks for index, URLs and citations
% (now we can use the command \url{...} to create URL with hyperlink)
% 
%\usepackage{color}
%\usepackage[a4paper,colorlinks=true,urlcolor=blue,citecolor=blue,linkcolor=blue,bookmarks=false]{hyperref}
%
% This allows inclusion of pictures.
% Create figures with PowerPoint and then export them individually
% in PDF, PNG, JPEG, or GIF format (in order of preference)
%
%\usepackage[pdftex]{graphicx}
%\DeclareGraphicsExtensions{.pdf,.png,.jpg,.gif}
%
% For pasting text files
%
%\usepackage{fancyvrb}
%
% Used to express formulas like n^th
%
%\usepackage{mathtools}
%
% For table formatting
%
%\usepackage{longtable}
%\usepackage{makecell}
%\usepackage{multirow}
%
% For plotting results
%
%\usepackage{pgfplots}
%\pgfplotsset{compat=newest}
%
% For placing floats
%
%\usepackage{placeins}
%
% For Appendix section
%
%\usepackage[titletoc,toc,title]{appendix}
%
% Definition of margins
%
%\usepackage[top=2cm,bottom=2cm,left=2cm,right=2cm]{geometry}
%
% Paragraph skip and indent
%
%\setlength\parskip{\medskipamount}
%\setlength\parindent{0pt}
%
% For itemize and enumerate spacing
%
\usepackage{enumitem}
% 
% For International System of Units (SI)
%
%\usepackage[binary-units]{siunitx}
%\sisetup{per-mode=symbol} % 1/s instead of s^-1
%
% For programming code
%
\usepackage{listings}
\lstset{
basicstyle=\ttfamily,
columns=fullflexible,
xleftmargin=3ex,
breaklines,
breakatwhitespace,
escapechar=`
}

% Some page parameters
\setlength{\parskip}{\medskipamount}
% Horizontal rule
\newcommand{\HRule}{\rule{\linewidth}{0.2mm}}


%
% Frequently used abbreviations.
% - example: \ie this is an example
%
\def\eg{e.g.\xspace}
\def\ie{i.e.\xspace}
\def\chap{Chapter\xspace}
\def\sec{Section\xspace}
\def\myfig#1{Fig.~#1\xspace} % usage: \myfig{\ref{fig:tag}}
\def\mytab#1{Tab.~#1\xspace}
\newcommand{\ltx}{\LaTeX\xspace}
\newcommand{\txw}{TeXworks\xspace}
\newcommand{\mik}{MikTex\xspace}
\newcommand{\html}{HTML\xspace}
\newcommand{\xhtml}{XHTML\xspace}
\newcommand{\cmd}[1]{\texttt{#1}\xspace}

% Styles
\newcommand{\itemname}{\textbf}
\newcommand{\thead}{\textbf}

% To cite RFC, es. \rfc{822}
\newcommand{\rfc}[1]{RFC-#1\xspace}
% To cite file (es. \file{autoexec.bat}) or fake URI (i.e. \file{http://www.lioy.it/})
% for real URIs use \url o \href
\newcommand{\file}[1]{\texttt{#1}\xspace}
% For inline code
\newcommand{\code}[1]{\lstinline|#1|}
% Backslash
\newcommand{\bs}{\textbackslash}
% Term definition with insertion into the index
\newcommand{\tdef}[1]{\textit{#1}\index{#1}}
% Meta-term
\newcommand{\meta}[1]{\textit{#1}}


\begin{document}
\selectlanguage{english}

\CorsoDiLaureaIn{Corso di Laurea Magistrale in }
\TesiDiLaurea{Master Thesis}
\AdvisorName{Supervisor}
\CandidateName{Candidate}


\logosede{res/logopolito}
\ateneo{Politecnico di Torino}

\titolo{Pin Control Defense}
\sottotitolo{Protecting Linux Kernel Pin Control Subsystem from Pin Control Attack}

\corsodilaurea{Ingegneria Informatica}

\candidato{Andrea \textsc{Genuise}}

\relatore{prof.\ Antonio Lioy}

\sedutadilaurea{\textsc{Academic~Year} 2016-2017}


\errorcontextlines=9

\frontespizio
\paginavuota
\newpage

\advance\voffset -5mm
\advance\textheight 30mm


\begin{dedica}
\textdagger\ A nonno Carmelo
\end{dedica}


\sommario

Recently embedded systems have become more and more integrated with all aspects of our lives, and their security concerns have risen as well.
They spread in various fields such as automotive, electronic devices, home automation, manufacturing and mission critical applications. These systems, in particular PLCs
deployed within the context of an Industrial Control System, use Input/Output interfaces to interact with the physical world by means of sensors and actuators.
As demonstrated by a novel kind of attack called Pin Control Attack, one can tamper with the integrity or the availability of legitimate I/O operations,
factually changing how a PLC interacts with the outside world and possibly causing physical damage to people and environment.
In this thesis we design a possible countermeasure to the attack and implement it for Linux Kernel on an ARM-based Programmable Logic Controller,
showing its effectiveness and impact on PLCs which usually have very limited resources and strict timing requirements.


\ringraziamenti

TODO Acknowledgements.


%\tablespagetrue
%\figurespagetrue

\indici

\mainmatter


\chapter{Introduction}
\label{chap:intro}

Embedded systems are widely used today in various applications, from consumer such as cars, cell phones, home automation, to critical infrastructures
like power plants and power grids, water, gas or electricity distribution systems and production systems for food and other products.
Within the context of an Industrial Control System (ICS), these systems are known as PLCs (Programmable Logic Controllers).

But security for these systems is an open question and could be a more difficult long-term problem than security for desktop and enterprise computing,
both for their limited capabilities and for the physical side effects a security breach could lead to, including property damage, personal injury, death and
even environmental or nuclear disaster.

The PLCs control the outside world via their I/O interfaces: therefore, they must be both reliable and secure.
Digging into their architecture, we know that the I/O interfaces of PLCs (e.g., GPIO, SCI, JTAG, etc.),
are usually controlled by a so called System on a Chip (SoC), an integrated circuit that combines multiple I/O interfaces.
In turn, the pins in a SoC are managed by a pin controller, a subsystem of SoC, through which one can configure the operating mode of the pins, such as the input or output mode.
One of the most peculiar aspects of a pin controller is that its behavior is determined by a set of registers: by altering these registers one can change the behavior
of the chip in a dramatic way. In \cite{ghostplc}, Abbasi et al. showed how this feature is exploitable by attackers, who can tamper with
the integrity or the availability of legitimate I/O operations, factually changing how a PLC interacts with the outside world.

Based on these observations, they introduced a novel attack technique against PLCs, called Pin Control Attack.
The salient features of this new class of attacks are:
\begin{enumerate}
	\item It is intrinsically stealth. The alteration of the pin configuration does not generate any interrupt, preventing the Operating System (OS) to react to it.
	\item It is entirely different in execution from traditional techniques such as manipulation of kernel structures or system call hooking, which are typically
		monitored by anti-rootkit protection systems.
	\item It is viable. It is possible to build concrete attack using it.
\end{enumerate}

To overcome this sophisticated attack, we propose a defensive mechanism to extend the existing Linux Kernel Pin Control Subsystem to be able to detect our novel attack.
It is a challenging task for two reasons:
\begin{enumerate}
	\item The Pin Control lacks hardware interrupts, so is not possible to directly react to any configuration change. More complex detection mechanisms are needed
		in order to achieve the highest possible detection rate.
	\item The resources available within an embedded system like a PLC are extremely limited. Therefore, our solution must be extremely agile and light
		since the smallest delay in the PLC I/O operation can have unintended consequences for the controlled process.
\end{enumerate}

In \chap \ref{chap:related} we show how Pin Control Attack is different from the majority of attacks in the literature, and discuss the protection mechanisms currently available
for embedded systems, showing that no one of these is actually capable of detecting the attack.
In \chap \ref{chap:design} we present the architecture of our proposed Pin Control Protection, then we describe the implemented modules and provide an user manual.
The \chap \ref{chap:results} provides the results obtained during the experiments, showing the detection rate and the performance overhead on a PLC environment.
Finally, in \chap \ref{chap:conclusions} we analyse the shortcomings of our defense and the possible future works and improvements.


\chapter{Related work}
\label{chap:related}

In this chapter we summarise the state of the art about security of embedded systems at the time of writing.
First, we discuss about the attacks of the recent years, showing how the embedded systems security concerns are rising.
Next, we analyse the defense mechanisms currently available in the literature, realising that they are still in a very early stage of their life.

\section{Attacks}

In the past few years a lot of attacks targeted embedded systems: most notably the infamous Stuxnet \cite{stuxnet} targeting an Iranian nuclear facility in 2010.
More recently Grandgenett et al. \cite{io-command} analysed the authentication protocol between the RSLogix 5000 software and the PLC, based on a simple challenge-response mechanism.
Since the protocol lacks freshness in its messages, is vulnerable to replay attacks, through which an attacker could repeat privileged commands to the PLC.
Furthermore, they found that both the decoding of the challenge and the encoding of the response use an RSA-2048 key which is hard-coded in the RSLogix software,
and it is actually valid for any Rockwell/Allen Bradley PLC.
This indicates how the security mechanisms of these systems often have a really poor design, if any.

Papp et al. \cite{taxonomy} analysed the existing attacks on embedded systems, relying on the proceedings of security conferences, with a focus on computer hacking,
and on the Internet for media reports, blogs and mailing list. They built a taxonomy based on five dimensions: precondition, vulnerability, target, attack method and effect of the attack.

For our purpose, we may classify the attacks found in literature using a simpler criterion based on the attack method. We may distinguish three main categories:
\begin{enumerate}
	\item \itemname{Firmware modification}: all the attacks aiming to upload a malicious firmware version (or part of it) in the device belong to this category.
	\item \itemname{Logic modification}: this category consists of the attacks that modify the PLC logic in some way. In this case the integrity of the firmware is not violated,
		but a malicious program, or logic, is inserted into the PLC to make it misbehave during the control process.
	\item \itemname{Control flow modification}: it includes the attacks that alter the normal control flow of a process by leveraging classic programming vulnerabilities
		such as buffer overflow or expired pointer dereference.
\end{enumerate}

We briefly report about these different kind of attacks in the following sections.


\subsection{Firmware modification attacks}

Almost all modern embedded systems provide a way to update the firmware, and the attackers could exploit this feature to upload its own malicious firmware.
Basnight et al \cite{firmware-mod} reverse engineered an Allen-Bradley ControlLogix L61 PLC firmware showing how to bypass the
firmware update validation method and successfully upload a counterfeit firmware.
Peck et al. \cite{ethernet-vuln} demonstrate how using commonly available software an attacker can write and load his malicious firmware into Ethernet cards of devices
used in control systems, potentially infecting the entire industrial control system.
Cui et al. \cite{print-vuln} discovered a vulnerability in the HP-RFU (Remote Firmware Update) feature of LaserJet printers,
that allows remote attackers to make persistent modifications to the printer's firmware by simply printing to it.


\subsection{Logic modification attacks}

Stuxnet \cite{stuxnet} belongs to this category. Along with its several components, mainly used to replicate and control the malware,
its core is essentially an infected version of a SCADA software library used to program the PLC itself.
By hooking some of the library functions it is able to load infected code and data blocks into the PLC and hide itself from the operator.
McLaughlin et al. \cite{dynamic-payload,sabot} evaluated some techniques and implemented a tool (SABOT) to infer the structure of a physical plant and craft a dynamic payload,
allowing an attacker to cause an unsafe behavior without having a deep \emph{a priori} knowledge of the target physical process.
Similar techniques might mitigate the precondition needed by an attack, making it even more viable.
Beresford \cite{siemens-s7} showed how the PLCs and the protocols used for communication in control systems were built without any security in mind,
and demonstrated that they are affected by many vulnerabilities which may also enable the attacker to know the current configuration and rewrite the PLC logic.
More recently, Klick et al. \cite{plc-network} used an internet-facing PLC as a network gateway by prepending a backdoor, made of a port scanner and a SOCKS proxy,
to the existing logic code of the PLC.


\subsection{Control flow modification}

Many recent advisories \cite{schneider-bof,rockwell-vuln,rockwell-vuln2,elcsoft-vuln} from ICS-CERT (Industrial Control System Cyber Emergency Response Team)
report about various programming vulnerabilities affecting both PLC firmwares and control softwares. Most of them allow remote code execution and could be exploited
without requiring particularly high skills.
The vulnerabilities discovered by Beresford \cite{siemens-s7} also allow the attacker to insert a payload into the logic and subvert the control flow to execute
malicious code. Nevertheless, the majority of the PLCs run the applications with root privileges, so it is quite simple for an attacker to get a root shell.
One of the most dangerous kind of control flow attacks consists of ROP (Return-Oriented Programming) techniques, or similar variants \cite{jop,no-ret}
which leverage different sequence of instructions equivalent to a return instruction.
Since code vulnerabilities may affect embedded systems \cite{schneider-bof,rockwell-vuln,rockwell-vuln2,elcsoft-vuln,siemens-s7}, ROP techniques
are applicable as well. Furthermore, due to the limitations imposed by these systems, is even more challenging to defend against them.


\section{Defenses}




\chapter{Pin Control Attack}
\label{chap:attack}

Before describing Pin Control Attack, a deeper analysis of the architecture of the target system is needed.
Note that, although this paper is focused on PLCs, the architecture described in the next section is still valid for almost any embedded system.
After having a proper knowledge of the underlying architecture, we can go deep into the description of the attack.
We consider the idea behind the attack, showing how it is able to evade the currently available detection mechanisms.
Next, we go further extending the applicability to a real Programmable Logic Controller, describing its architecture in more detail.
Based on our architecture analysis and our tests, we can finally demonstrate that the attack is actually viable on real PLCs, even on a higher abstraction level.


\section{Embedded architecture}
\label{sec:embed_arch}

As briefly reported into \chap \ref{chap:intro}, PLCs use I/O interfaces to communicate with sensors and actuators, and in general with any external device.
Digging into their architecture, we know that PLCs are usually based on a so called System on Chip (SoC).
A SoC is basically an integrated circuit made of a microprocessor, a memory block and a set of peripheral controllers all enclosed together in the same chip substrate.
Thus, the SoC technology provides fully capable computers having both very small size and low power-consumption.
A SoC usually comes with a set of connections, also known as \emph{pins}, usually soldered to a printed circuit board to facilitate interconnection with external modules.
Many types of pins may be included in a SoC, having different purposes (power, clock, I/O, etc.).
An example of such a system is the Raspberry Pi board shown in \myfig{\ref{fig:raspberry}}, based on a Broadcom System on Chip.
\begin{figure}[h]
\centerline{\includegraphics[width=0.56\textwidth]{res/raspberry}}
\caption{Raspberry Pi \cite{raspberry} with Broadcom System on Chip \label{fig:raspberry}}
\end{figure}
Actually almost all of the embedded systems use a SoC with similar boards, each one with its own size and configuration.

In order to accommodate different potential implementations, each pin (or group of pins) of a SoC may have multiple configuration and operating modes,
depending on the board they are attached to. The configuration of the pins is managed by the pin controller, a particular subsystem of any SoC.
Through the pin controller, the system can configure the operating mode of the pins, such as their input or output mode.
The features of a pin controller can be grouped into two main categories:
\begin{itemize}
	\item pin configuration: allows the system to change some electrical properties of the pins, such as direction, event detect, interrupt, etc.;
	\item pin multiplexing: each pin of the SoC may have many usage usages, also known as \emph{alternate functions}, depending on what is needed by the external board.
		The pin multiplexing feature enables the system to specify which type of function is needed on each pin.
\end{itemize}
As the I/O attack is a very low-level attack, it is necessary to dig further into the electrical world to know how these I/O interfaces work.


\subsection{SoC pins}
\label{sec:iopins}

I/O interfaces of a System on Chip provide a connection between internal modules and external electronic devices.
As shown in \myfig{\ref{fig:chips}}, they are physically visible from the outside of the chip package,
usually in the form of pins \subref{fig:pins} or soldering balls \subref{fig:balls}.

\begin{figure}[h]
\centering

\begin{subfigure}{.45\textwidth}
\centering
\includegraphics[width=\linewidth]{res/pins}
\caption{\label{fig:pins}}
\end{subfigure}
\begin{subfigure}{.45\textwidth}
\centering
\includegraphics[width=\linewidth]{res/balls}
\caption{\label{fig:balls}}
\end{subfigure}

\caption{I/O connections packaged as \subref{fig:pins} Pin Grid Array and \subref{fig:balls} Ball Grid Array\label{fig:chips}}
\end{figure}

Internally they are connected to the silicon die through bonding wires, and are managed by a specific I/O circuit which may vary according to the specific chip.
Although there are many different implementations, almost all of the available SoCs have I/O ports with very similar functionalities.
For our purpose, we can describe them in an implementation-independent manner by using simplified generic schematics.


\subsubsection{Pin configuration}
\label{sec:pinconf}

The schematic depicted in \myfig{\ref{fig:pinconf}} helps us to discuss about the first set of features: pin configuration.
The operation mode described in this section is also known as General Purpose I/O (GPIO).

\begin{figure}[h]
\centerline{\includegraphics[width=0.8\textwidth]{res/pinconf}}
\caption{General Purpose I/O pin configuration circuit \label{fig:pinconf}}
\end{figure}

Apart from the protection diodes that serve as shield against input currents lower than $V_{SS}$ or higher than $V_{DD}$,
the circuit is divided into two different parts: one for output and one for input.
\begin{itemize}
	\item \itemname{Output driver}:
		The output module is basically a buffer controlled by an output enable signal.
		This signal controls the direction of the pin (input or output). If it is high, then the pin is in output mode
		and the value coming from a write operation goes through the buffer to the actual pin.
		If it is low, the pin is in input mode and the write signal is blocked, so it is not possible to change the external value of the pin from inside anymore.
	\item \itemname{Input driver}:
		The input driver has a similar buffer to read the value, usually having hysteresis capability which is useful for filtering unstable external values.
		Since the read buffer is always active, the input value is always available, even if the pin is currently working in output mode.
		The reason for this is merely physical: even if the external value was blocked by the buffer,
		one would always get a value by reading the input signal, because a value is nothing but an interpretation of the voltage level on a wire.
		When the pin is set as input, the pull-up/pull-down network enables the user to have a ``default'' value on the pin,
		namely a defined state maintained while the pin is not actively driven from outside. This feature is useful to avoid so called ``floating'' inputs.
\end{itemize}

For Pin Control Attack, what is more interesting about the circuit in \myfig{\ref{fig:pinconf}} are the following two properties:
\begin{itemize}
	\item there is no checking about the input state, so it is possible to perform a read even when the pin is in output mode;
	\item vice versa, it is not possible to write to a pin which has been configured as output.
\end{itemize}

Furthermore, it is also possible to drive the pull-up/down network, factually disturbing the real value of the I/O pin in an unpredictable way.
In this case the effects strongly depends on the actual implementation of the printed circuit board as well as on the external components connected to the I/O pin.


\subsubsection{Pin multiplexing}

Inside the chip an I/O pin may be connected to more than one device, which can be selected depending on the application,
that is the way of soldering or wiring the package into an electronic board.
This SoC feature is known as pin multiplexing (also ball multiplexing, pad multiplexing, alternate functions).
Even though pin multiplexing is designed for hardware configuration, in almost all modern chips it is possible to change the function at runtime.

\begin{figure}[h]
\centerline{\includegraphics[width=0.8\textwidth]{res/pinmux}}
\caption{Generic I/O pin multiplexing circuit \label{fig:pinmux}}
\end{figure}

\myfig{\ref{fig:pinmux}}~ shows a possible hardware implementation of pin multiplexing.
The I/O pin in the figure is connected to two different peripherals inside the chip, namely A and B,
and it is also accessible through basic GPIO as described in \sec \ref{sec:pinconf} above.
The access to the GPIO output driver is regulated by two in cascade multiplexers for each signal of the module.
If the peripheral A is enabled, then the signals driven by A go through the multiplexers and can drive the actual value of the pin,
while GPIO and peripheral B output signals are blocked. Instead, if only peripheral B is enabled then only its signals are able to reach the I/O pin.
Note that in this last case the peripheral A should not be enabled, because A multiplexers have precedence against B ones.
Thus, the cascading of multiplexers actually implements a priority mechanism between peripherals A and B.
If neither A nor B are enabled, then no alternate function is active for the I/O pin and it could be driven by GPIO signals.
Each peripheral may also have its own dedicated input line, to get values from the I/O port in the same way as GPIO does.

For our purpose, at least two interesting properties can be inferred from pin multiplexing schematic of \myfig{\ref{fig:pinmux}}:
\begin{itemize}
	\item it is possible to block output signals from peripherals by simply changing the multiplexing configuration;
	\item the GPIO value can be read at any time, independently from the current multiplexing state of the output.
\end{itemize}


\section{Pin Control Attack analysis}

Due to our previous discussion, we can argue that Pin Control Attack is actually applicable on any System on Chip having the
architecture described in \sec \ref{sec:embed_arch}. However, PLCs represent a particularly interesting target among all the embedded systems
based on SoC. PLCs leverage the interfaces of a SoC to interact with sensors and actuators, having direct effects on our physical world.
Therefore, if the I/O of a PLC is compromised, this constitutes a much more significant security and safety risk with respect to other
non-critical embedded system.

As described in \cite{ghostplc}, Pin Control Attack is designed to evade off-the-shelf host-based detecion systems.
Furthermore, differently from other attacks, its target is neither the firmware nor the logic of the PLC.
The idea of Pin Control Attack is that the attacker can operate at the lowest level possible, targeting the interaction
between the firmware and the PLC I/O, as shown in \myfig{\ref{fig:target}}. For this reason it is also known as I/O attack.
Despite its crucial function in embedded systems, the I/O hardware architecture and the I/O drivers are basically designed
without any concept of security, assuming that I/O is reliable. Unfortunately, given the properties inferred in \sec \ref{sec:embed_arch}
from generic hardware architecture, this is often not the case.

\begin{figure}[h]
\centerline{\includegraphics[page=6,viewport=50 620 300 750,clip]{res/ghostplc}}
\caption{Pin Control Attack target, from \cite{ghostplc} \label{fig:target}}
\end{figure}

In the following section we focus on the design of I/O attack. First, we summarise the host-based detection mechanisms taken into account
during the attack design and the techniques used to evade them. Second, we analyse the attack itself in more detail.
After the design part, we describe different implementations of the attack coming from our experiments,
showing how the attack works in practice. The goal of this chapter is to help us having a better understanding of the attack,
and to provide a detailed threat model on which our defense design will be based on.


\subsection{Design}

Based on what is available in literature, Pin Control Attack has been designed to be different from previous attack techniques and to
circumvent the off-the-shelf host-based detection systems. The authors of I/O attack have identified two main defensive mechanisms whose properties
makes them more easily applicable to PLCs. We briefly analyse these defenses and show how they can be circumvented by Pin Control Attack,
and then we describe the Pin Control Attack technique. Our purpose here is to have basic definitions of the concepts used by the attack,
see \cite{ghostplc} for a more detailed description.


\subsubsection{Defense analysis}

TODO Defense analysis


\subsubsection{Attack technique}

TODO Attack technique


\subsection{Implementation}

TODO Implementation.


\chapter{Pin Control Defense}
\label{chap:defense}

This chapter proposes a possible countermeasure to Pin Control Attack.
First, it describes the overall design of Pin Control Defense, showing different approaches to tackle the attack at different levels
and explaining our design choices. Second, it contains a detailed report of the defense implementation.


\section{Design}
We report the design phase below, starting from a preliminary analysis of the problem to a discussion of different possible solutions.
Then, a detailed description of the defense architecture concludes the section.


\subsection{Preliminary analysis}
\label{sec:pre_analysis}

The main goal of our defense is to detect and hinder Pin Control Attack while being able to not consume the limited resources inside the target system,
and to not affect real-time operations. The timing of the operations is fundamental in systems like PLCs, on which the minimum delay may alter the physical process.

We make the same assumptions contained in the threat model of \mysec{sec:threat_model}.
Therefore, we assume that the system, and our defense as well, is protected by the previously discussed HIDSs.
Starting from the analysis reported in \mychap{chap:attack}, we identify the I/O configuration as the main resource we aim to protect from Pin Control Attack.
To target I/O configuration, the attacker may use other system capabilities, such as virtual address mapping and debug registers.
First of all, we define the following components of a SoC, to which we refer several times in the rest of the paper:
\begin{itemize}
	\item \itemname{I/O subsystem}: the subsystem that controls the I/O configuration. I/O configuration is defined as the set of all the control registers
		actively used by the SoC, whose unathorized modification may have direct or indirect effects on the controlled process.
		In particular, we are interested into pin control registers, which directly determine the behaviour of the SoC I/O pins.
		However, a SoC typically contains many devices and controllers which may be in charge of a subset of I/O pins. In other words, some pins
		can be multiplexed to specifical devices inside the SoC. Since these devices are programmed through their own control registers,
		altering these registers may indirectly affect I/O pins as well. Thus, both pin control registers and device control registers
		are considered as part of the I/O configuration. The attacker who has knowledge of the system and its I/O peripherals may access
		all these registers to alter the physical process.
	\item \itemname{Debug subsystem}: the SoC subsystem that enables debug capabilities for the operating system and its processes.
		Typically, it consists of a set of registers, called \emph{debug registers}, inside the processor of the SoC.
		The operating system may provide an interface to access them, both for kernel side and user side.
		The attacker may leverage the debug subsystem to obtain accurate timing information and conduct more sophisticated attacks,
		either using the interface provided by the operating system or the low-level processor instructions directly (depending on its privilege level).
	\item \itemname{Mapping subsystem}: the system that manages mappings between physical and virtual addresses, both for kernel and user space.
		It is typically supported by the Memory Management Unit (MMU) in hardware, and by the operating system in software. Within the context of a PLC,
		the runtime may use this subsystem to configure the I/O and to perform read/write operations. An attacker can access physical I/O addresses
		(thus, I/O configuration) either by requesting a new mapping to the system or by using an already existing mapping.
\end{itemize}

Given the architecture described in \mysec{sec:embed_arch}, protecting I/O configuration is not straightforward because the hardware lacks
any protection mechanism related to the I/O subsystem. For instance, the SoC might generate a trap for each modification to I/O registers.
However, this approach would not be reasonable, because of the huge overhead imposed to any I/O access.
Alternatively, a trap could be generated only when an I/O access is malicious. For example, to prevent pin configuration attack,
a trap signal might be delivered to the CPU when the internal signal of a pin is different from the external one, meaning that the pin is misconfigured.
A similar approach may be used for pin multiplexing, by checking the internal signal connected to the multiplexed device.
Unfortunately, since these approaches would require significant hardware modifications, with subsequent updates of all the SoC drivers, they are very unlikely to be applied.

Thus, we need to define an alternative approach that does not have the above limitations: it must be easily applicable and have a minimal overhead.
Since the hardware-based solution is unpractical, we analysed the possible software-based solutions.
In order to choose the best strategy against Pin Control Attack, we compared the following approaches, as proposed in \cite{ghostplc}:
\begin{enumerate}
	\item monitoring I/O configuration to detect changes;
	\item monitoring the use of debug registers;
	\item monitoring the mapping requests targeting I/O configuration;
	\item monitoring performance overhead;
	\item using a trusted execution environment.
\end{enumerate}
These approaches are not mutually exclusive, and may be used simultaneously to get a higher protection level.
In this work, we decided to design and implement the first three approaches in combination, because they directly protect the three different resources that the attacker may use.
Thus, they represent a good minimum set of countermeasures capable of raising the bar for the attacker (see \mysec{sec:def_arch}).
Our solution, however, may be extended to include the last two approaches as well, to cover some existing limitations (see \mysec{sec:def_sec}).
Before describing our design phase, we briefly discuss these two further protections in the following sections.


\subsubsection{Monitoring performance overhead}

A detection system based on performance monitoring may be useful to add a further protection, especially against the attack variant
which uses debug registers. In general, such a monitor would be also useful for other kind of attacks targeting PLCs, because these systems cyclically performs
a limited amount of well-known operations. This simplifies the detection of any deviation from the standard behaviour.
In our case, a performance monitor would be helpful to cover those attacks that the first three strategies are not able to detect,
although, as discussed in the following sections, those are very limited cases.
Deploying a performance monitoring system, anyway, does not require very high effort because it can leverage \emph{Hardware Performance Counters} (HPC),
nowadays available in almost all the SoC processors. The most challenging part would be, of course, integrating the monitor with the PLC runtime.
When a new logic is uploaded to the PLC, a burst of operations are executed by the runtime to check the new logic code, apply the new configuration and start the logic.
These extra operations, of course, must not be labeled as malicious. Furthermore, once a new logic started, it may perform operations that are different
with respect to the previous one; therefore, the monitor should be able to recognise this change and update its statistics as well.


\subsubsection{Trusted execution environment}

A Trusted Execution Environment (TEE) is a particular set of hardware and software components providing security features,
such as isolated execution, integrity of Trusted Applications (TAs), and integrity and confidentiality of TAs assets \cite{tee}.
The TEE technology is based on the concept of partitioning a computing system into Secure and non-Secure world, where
the code running into non-Secure world cannot access the Secure partition.
A lot of effort has been put into the standardisation of the TEE, and some commercial solutions already exist,
such as Intel Trusted Execution Technology (TXT) \cite{intel_txt} and ARM TrustZone \cite{trustzone}.
Since embedded systems are our main target, we provide a brief description focused on the ARM TrustZone implementation.
In this technology, the physical memory is partitioned into Secure and non-Secure regions, and the processor core is divided into Secure and Non-secure virtual cores.
The virtual core is distinguished by the NSTID (Non-Secure Table IDentifier) bit associated with the current instruction,
while each instruction or data address (\ie each bus transaction) is marked with an NS bit.
The protection of the Secure world is guaranteed by checking all the accesses to memory or peripherals.
The Non-secure core can only access Non-secure memory regions, while the Secure world can use both Secure and Non-secure addresses.
This partitioning is parallel and independent from Supervisor/User modes available on the CPU. Therefore, each world has its own supervisor and user mode as well.
To improve the performance of this architecture, TLBs and caches may support the NS attribute for each entry as well.
This enables Secure and Non-secure entries to co-exist avoiding TLB and cache flushes on every switch between the two worlds.
The switching between Secure and Non-secure world is managed through a specific Secure Monitor Call (SMC) instruction,
which changes the core mode into Monitor Mode. The code executed into monitor mode is always Secure, and it basically performs the context switch
between the two virtual cores. If a Secure process is loaded, NSTID bit is set accordingly.
Since the Secure world may contain a whole parallel micro-kernel with trusted user applications, it may be possible
to deploy Pin Control Defense inside the trusted domain, protecting the defense itself from defense-aware attackers.
However, the overhead imposed by a trusted execution environment may be unacceptable for embedded systems with real-time constraints like PLCs.
Hence, the impact of this solution still needs to be investigated.


\subsection{Defense Architecture}
\label{sec:def_arch}

Based on the previous considerations, we designed a detection system which is able to protect against Pin Control Attack at three different levels,
corresponding to the SoC I/O, debug and map subsystems. The system is designed to run as part of the operating system, thus having kernel privilege level.
Its overall architecture is shown in \myfig{fig:defense}.
\begin{figure}[h]
\centerline{\includegraphics[width=0.8\textwidth]{res/defense}}
\caption{Pin Control Defense general architecture \label{fig:defense}}
\end{figure}
Considering that the attacker can possibly follow any of the paths highlighted in red in the figure,
our monitoring system has been divided into three main components:
\begin{itemize}
	\item \itemname{I/O monitor}: its purpose is to watch the I/O configuration, detect and react to any malicious change.
	\item \itemname{Debug monitor}: it aims to protect the debug subsystem from malicious usage.
	\item \itemname{Map monitor}: it acts as a filter for mapping requests targeting I/O memory.
\end{itemize}

Each monitor is responsible for reporting detection information related to any interesting event of the respective subsystem,
and reacting to these events according to its own configuration. The events are not necessarily due to Pin Control Attack,
\eg an I/O configuration change event may be caused by the PLC runtime.
Therefore, the I/O monitor should decide whether a particular I/O modification can be considered legitimate or not.
Debug and map monitors, instead, serve at least the following purposes:
\begin{itemize}
	\item \itemname{early detection}: the attack can be detected before it modifies the I/O configuration;
	\item \itemname{raising the bar}: they limit the attacker possibilities, by restricting the access to the corresponding subsystem;
	\item \itemname{reporting info}: they provide additional information, useful to figure out how an eventual attack has been conducted.
\end{itemize}
Since these modules are designed to be customisable, the actual effects depends on their configuration.
The following sections describe in more detail the design and the role of each module.


\subsection{I/O monitor}
\label{sec:io_design}

The I/O monitor plays the most critical role into our detection system.
Its main task is to cyclically check the values contained into I/O configuration registers, to verify that they are conforming with the logic currently loaded into the PLC.
If a change into a target register has been detected, it determines whether this modification is legal or not, according to a given trusted behaviour.
As discussed in \mysec{sec:plc_arch}, a new I/O configuration may come with a new PLC logic, and the PLC runtime must be able to apply the change without
having our defense to interfere.
Therefore, an automated mechanism able to distinguish between trusted and malicious configurations is needed.
This is not an easy problem, because an optimal solution would require an authentication between PLC runtime and I/O subsystem,
and this is not reasonable in our highly constrained system.

To tackle this problem, we designed the following strategy, which we used to implement the I/O monitor:
\begin{enumerate}
	\item define I/O configuration registers;
	\item define a trusted behavioural model of the I/O configuration;
	\item \label{enum:io_monitor} constantly monitor the I/O configuration to detect possible modifications;
	\item if a change is detected, verify whether it is conforming to the defined behaviour or not;
	\item if it is, accept the new configuration and go back to \ref{enum:io_monitor};
	\item otherwise, it is probably Pin Control Attack: react according to the configured monitor action and go back to \ref{enum:io_monitor}.
\end{enumerate}
The first step is to choose which registers should be included into the I/O configuration, \ie which registers should be protected.
This set of registers includes, ideally, all the registers whose modification may produce an effect on the physical process.
However, it is not always possible to define a behavioural model or enable the protection for each register. For instance,
some registers may be write-only (\eg pull-up/down registers), and there is no way to verify their current value.
Whether to include or not a register into I/O configuration should be decided case by case,
according to the protection feasibility and the possible effects of a malicious modification.

To define the behavioural model, it is required to determine the set of configuration registers actively used into the target system,
\ie the set of registers that may affect the physical process. Then, for each register (or for each bit of each register if necessary),
the model should define under which conditions the corresponding value may change. These conditions are highly dependent on the target implementation.
Generally speaking, they can be represented by simple time constraints (\eg the value cannot change twice within $\SI{20}{ms}$),
logical conditions (\eg the value must be conforming to the running PLC logic), a statistical model, or a combination of these.
Each condition can either provide an exact distinction between a trustworthy and a malicious modification (typically logical conditions),
or a heuristic only. For this purpose, we analyse in more detail the proposed logical condition, since it is able to completely exclude false positives.
The condition states that a change can be accepted only if the new I/O configuration is in line with the operations performed by the running PLC logic.
Checking this condition is feasible as long as enough knowledge of the PLC runtime is available. To obtain this knowledge, two ways are doable:
\begin{itemize}
	\item \itemname{reverse engineering}: by analysing the PLC software it is possible to dynamically obtain the required information from the running logic;
	\item \itemname{PLC runtime vendor collaboration}: if the PLC software is designed to be aware of the defense mechanism,
		it could better expose the required information to the I/O monitor at run-time, thus removing the need for reverse engineering.
\end{itemize}
Once the I/O monitor is able to determine which operations the PLC logic is carrying on, it can easily decide if an I/O configuration is malicious or not,
and can effectively detect Pin Control Attack (see our approach in \mysec{sec:io_impl}).
Note that, if the information on the current I/O operations is made available by the PLC runtime, the attacker can modify it as well while conducting the attack.
In this case, however, the attacker needs to modify the PLC logic, and this may be easily noticed if further protection mechanisms are provided by the PLC runtime.
Therefore, the attack would lose one of the features that made it stealth.

An important parameter to discuss is the time interval of the main monitor loop. Since no other mechanisms are provided by the hardware,
such as interrupts, we can only detect I/O configuration changes by cyclically checking its current values. Greater scanning intervals may give the attacker
enough time-window to reach its purpose. For instance, given a PLC scan cycle of $\SI{10}{ms}$ and a monitor interval of $\SI{50}{ms}$,
if the attack is able to synchronise itself with the monitor loop, it has enough time to alter $\left \lfloor{\frac{\SI{50}{ms}}{\SI{10}{ms}}}\right \rfloor = 5$
consecutive I/O operations and then restore the configuration back before getting noticed. Smaller intervals, instead, may cause too much performance overhead.
We provide our experimental results in \mychap{chap:results}.

Another challenging aspect of this approach is to decide which action the monitor should follow if an attack has been detected.
We distinguished at least three main reactions for I/O monitor:
\begin{itemize}
	\item report the event to the system;
	\item revert the configuration back to the last known before the attack;
	\item stop the control process.
\end{itemize}
The choice among these actions (or a combination of them) again depends on the target implementation.
Typically, reporting the event is the minimum that the monitor can do, while other reactions should be decided according to risk associated with the physical process
and to the monitor reliability. If a detection is proven to be correct, due to an exact condition,
then reverting the configuration back could be the best choice. Otherwise, if the detection condition is a heuristic and the risk is critical,
stopping the control process may be considered as a more cautious alternative.
In general, if a monitor only reports about events we call it \emph{passive}, otherwise it is an \emph{active} monitor.


\subsection{Debug monitor}
\label{sec:dr_design}

This monitor is responsible for protecting the debug subsystem from malicious usage. Debug registers may be used by the attacker to gain accurate timing
information about the PLC logic I/O operations.
Similarly to what the I/O monitor does, the debug monitor continuously watches the values contained into debug registers to detect malicious modifications.
In our design, we assumed that there is no need to use debug registers into the target PLC if it is already deployed into a real control system.
As confirmed by our experiments, they are actually never used.
Typically, the SoC debug subsystem may only be needed by PLC vendors during design and implementation of their own product.
Since the operating system may provide user level access to debug registers (\eg \verb|ptrace| API on Linux), we can simply disable this user interface.
However, there is no mechanism to permanently disable debug registers also at kernel-wide level.
To understand this problem, we need to distinguish the following two cases: the debug support can be either enabled or disabled into the kernel itself.
If it is enabled, the attacker may simply leverage the system interface to use debug registers from kernel space.
If debug support is disabled, an attacker who gains kernel level access (as in kernel module version of Pin Control Attack) can always re-enable them at run-time
by inserting its own debug exception handler into the OS interrupt vector table (see \cite{arm-evt,x86-idt}).
However, this kind of attack is already covered by the defenses assumed in our threat model, because it is a data hooking technique.
Thus, we designed the following monitor strategy, that is required only if debug support is enabled into the operating system:
\begin{enumerate}
	\item disable debug registers user space interface;
	\item \label{enum:debug_monitor} constantly monitor debug registers to detect possible modifications (from kernel space);
	\item if a change is detected, it is Pin Control Attack: react according to the configured monitor action and go back to \ref{enum:debug_monitor}.
\end{enumerate}
The strategy is (in part) a simplification of the I/O monitor approach, on which the trusted behavioural model assumes that debug registers never change in a production system.
Based on this assumption, the debug module allows our defense to provide an early detection of the attack, before I/O configuration is actually altered.
The discussion about the monitor time interval is the same as for the previous I/O monitor: a trade-off between attacker time-window and monitor overhead.

When an attack has been detected, restoring the previous values of debug registers is surely the best action to take,
because the assumption ensures that only malicious changes may occur.
Halting the PLC process, instead, is certainly not needed because the control process is not directly affected by a modification of the debug subsystem.
In any case, the event is reported to the system. To uniform the design, the debug monitor can be configured either as passive or active,
although the passive mode is strongly discouraged for the above reasons. If the monitor is active, it actually raises the bar for the attacker,
who cannot leverage debug registers anymore.


\subsection{Mapping monitor}
\label{sec:map_design}

As discussed in \mysec{sec:pre_analysis}, the attacker may either request a new mapping between physical and virtual memory, or re-use an existing one before
modifying I/O configuration. The map monitor leverages this fact, providing a further detection mechanism usable in combination with the previous two.
Typically, the operating system provides an interface, for user space, through which each process can map a physical address region to a corresponding virtual region.
In the following part of this document we refer to it as ``mapping interface''.
The actual mapping is performed inside the kernel, and the process receives a valid virtual address as result.
The attacker can leverage this mechanism to gain access to the I/O configuration registers from user space.
According to the implementation, the PLC runtime may use this mechanism as well, and if it does, the attacker may try to re-use the PLC runtime virtual address.
We distinguish at least two techniques to re-use existing virtual addresses:
\begin{itemize}
	\item exploit a remote code execution vulnerability on the PLC runtime owning the addresses;
	\item set debug register on PLC runtime virtual address (the debug handler will have access to the process virtual addresses).
\end{itemize}
We exclude the second technique, because it is already detectable by the debug monitor. The first one, instead, can only be counteracted by detecting the control flow attack itself.
Thus, the aim of this monitor is not to detect addresses re-using, which is out of our scope, but to monitor \emph{new} mapping requests.
To achieve the goal, we dynamically replace the functions belonging to the mapping interface with our own versions (hook).
We can summarise the strategy of the map monitor as follows:
\begin{enumerate}
	\item \label{enum:map_model} define a trusted behavioural model for I/O mapping requests of the PLC runtime;
	\item hook all the functions belonging to the mapping interface;
	\item at each new mapping request, verify whether the requested physical address range overlaps the I/O configuration region or not;
	\item if there is no overlap, forward the request to the original system function;
	\item if an overlapping region is detected, verify whether the request is conforming to the trusted behaviour or not;
	\item if it is, forward the request to the original system function;
	\item if it is not, it is probably Pin Control Attack: react according to the configured monitor action.
\end{enumerate}
The behavioural model of step \ref{enum:map_model} should describe if and how the mapping interface is used by the PLC runtime.
We analysed the behaviour of our target systems, with the following results:
\begin{itemize}
	\item Raspberry Pi: the CODESYS runtime un-maps and re-maps the I/O every time a new PLC logic is uploaded;
	\item Wago PLC: e!RUNTIME never maps physical I/O from user space, because it is managed by the system driver.
\end{itemize}
The next step of the strategy, function hooking, must satisfy at least the following requirements (see implementation in \mysec{sec:def_impl} for more details):
\begin{itemize}
	\item it must be efficient: mapping functions may be called many times by processes (\eg in Linux, they are used not only to map physical memory,
		but any file, device, etc.);
	\item considering the threat model discussed in \mysec{sec:threat_model}, it must be applied before the Autoscopy Jr. detection system is deployed;
		otherwise, our modification will be considered as malicious.
\end{itemize}

Finally, the reaction of the map monitor to an eventual detection depends on the PLC runtime behaviour. If the PLC runtime maps the I/O (as in Raspberry Pi),
then a mechanism to distinguish between good and malicious requests is needed. To accomplish this, we may list the following alternatives:
\begin{itemize}
	\item heuristic approach based on statistical data;
	\item integration of the defense with the PLC runtime.
\end{itemize}
Since the first approach cannot give an exact detection, the monitor may simply report the detection to the system.
The second approach, instead, may be implemented in different ways. For instance, the map monitor could provide a separate mapping interface
reserved only for the PLC runtime process.
In any case, if the system allows to have an exact detection mechanism, the monitor may directly deny the malicious request, factually raising the bar for the attacker.


\section{Implementation}
\label{sec:def_impl}

We describe here our implementation of the above strategies, discussing the engineering problems encountered and the adopted solutions.
Since the authors of the attack presented their work as ``Ghost in the PLC'' \cite{ghostplc}, we called our defense prototype implementation \emph{Ghostbuster}.
Ghostbuster is a kernel module written in C language, targeting Embedded Linux running on ARM architecture.
It is designed to be highly configurable and as architecture-independent as possible, following the guideline of the Linux kernel itself.
A portion of the code, \ie the lowest level code, is still dependent from the specific architecture (\eg ARM), but is separated from the general implementation template.
This allows Ghostbuster to be easily extendable to other architectures and SoCs running Linux. At the same time, we focused on maintaining the lowest overhead possible,
which is always crucial for PLCs. We proceed with the description of the overall architecture, and then we go deep into each module of the architecture.
Finally, we describe the usage of our kernel module.


\subsection{Implementation architecture}

The implementation architecture is based on the general one described in \mysec{sec:def_arch}. Thus, we implemented I/O, debug and map monitors.
Each one of them has been divided into two main parts, following the template method pattern: the main strategy and the sub-actions implementation.
The aim of this separation is to minimise the effort needed to deploy our defense into different systems, thus improving its portability.
In particular, we considered the following variables: each target system may have its own System on Chip, firmware and PLC runtime.
During the design part, we defined high-level monitor strategies, which allowed us to provide an abstraction adaptable to any Linux-based system,
independently from these variables.
\begin{figure}[h]
\centerline{\includegraphics[width=\textwidth]{res/def_impl}}
\caption{Ghostbuster implementation architecture \label{fig:def_impl}}
\end{figure}
The resulting architecture is shown in \myfig{fig:def_impl}, which associates each specific role to the corresponding
source file(s) (in red).
Furthermore, the compilation of our module can be parameterised in different ways, as shown by the green labels in the figure.
Each green label, associated to a rectangle container in the figure, corresponds to a compiler flag that affects the compilation of the enclosed portion.
The compilation is managed by a \verb|Makefile|, in which all the required flags are defined.
The \verb|Makefile| needs to know the location of the target Linux kernel source directory.

In particular, each monitor can be enabled or disabled by means of the following flags:
\begin{Verbatim}[fontsize=\small]
	IO_MONITOR_ENABLED
	DR_MONITOR_ENABLED
	MAP_MONITOR_ENABLED
\end{Verbatim}
If a monitor is disabled, it will not be included into the compilation at all, reducing the final binary size.
This may be useful to exclude a particular monitor that is not needed by the target system
(\eg DR monitor if debug support is not enabled into the kernel, see discussion in \mysec{sec:dr_design}).

When a monitor is enabled, its execution mode may be controlled by one of the following flags, respectively:
\begin{Verbatim}[fontsize=\small]
	IO_MONITOR_ACTIVE
	DR_MONITOR_ACTIVE
	MAP_MONITOR_ACTIVE
\end{Verbatim}
If one of these flags is not defined, the corresponding monitor is compiled in \emph{passive} mode: it will only reports about events, without taking any further action.
Otherwise, the monitor is \emph{active} and it will include its specific reactions to the events.
If a monitor is disabled, the corresponding active flag is ignored.

The messages reported by each monitor may be extended by enabling the following debug flags:
\begin{Verbatim}[fontsize=\small]
	IO_DEBUG
	DR_DEBUG
	MAP_DEBUG
\end{Verbatim}
If a debug flag is defined, the corresponding monitor will print out its complete state at each event detection.

Finally, the \verb|ARCH| and \verb|SOC_MODEL| variables are used to select a specific implementation for the enabled monitors.
An implementation is identified by an architecture name and a SoC model (\eg \verb|arm|, \verb|BCM2835|). Given this information,
the implementation files will be automatically included from the \verb|<ARCH>/<SOC_MODEL>/| sub-directory.
We chose to directly include the implementation part into header files, instead of external compilation units (\verb|.c| files), to allow the usage of \verb|inline|
functions. The code of an inline function is directly included into the caller, and it does not need an explicit function call.
This may cause some code duplication, which we tried to minimise, but it achieves better performance,
especially for those functions that are called many times (\eg in the main loop of a monitor).

When Ghostbuster is loaded, the three main monitors are started in parallel.
The monitors are independent from each other, and every monitor has its own kernel thread.
When a monitor has to report an event to the system, it uses the common logging subsystem available into the kernel
(see \mysec{sec:def_sec} for a possible improvement on the reporting mechanism).
The target systems used to test our defense are exactly the same as the ones described for the attack part in \mychap{chap:attack}, with the same PLC logic and I/O configuration.
In particular, for each monitor, we first refer to the Raspberry Pi system (BCM2835 SoC); then we discuss the modifications required to run on Wago PLC as well.
In the following sections we describe in more detail the structure and the operations performed by each monitor.
From now on, we will refer to the abstract part of each monitor as the \emph{interface}, and to the architecture-dependent part as the \emph{implementation}.


\subsection{I/O monitor}
\label{sec:io_impl}

This monitor is responsible for protecting I/O configuration memory from malicious usage.
The monitor interface includes the I/O configuration data type, modeled as a set of memory blocks by means of the \verb|io_conf_t| structure:
\begin{lstlisting}
typedef struct {
	const void** addrs; // Set of block base addresses
	const unsigned* sizes; // Size of each block in bytes
	const unsigned blocks; // Number of blocks
	const unsigned size; // Total size in byte
} io_conf_t;
\end{lstlisting}
This model takes into account the fact that I/O registers may be located at different addresses, resulting in a non-contiguous I/O memory.

With reference to the BCM2835 manual \cite{bcm2835} and to the target configuration defined in \mysec{sec:attack_pi},
we defined, into the implementation part, the set of I/O registers we want to protect.
In particular, since our system uses GPIO pins, two of which multiplexed to I2C, we chose to protect GPIO pin control registers, whose physical base address is \verb|0x2020000|.
These registers are responsible both for pin configuration and pin multiplexing. Each register is 32-bit wide, having 3 bits for each I/O pin: it can control $10$ different pins.
Although our target system uses only four pins for its I/O, to be more general, we decided to protect all the available GPIO pins ($54$).
Hence, we included all $6$ pin control registers into the definition of our I/O configuration, by filling in the \verb|io_conf_t| global structure:
\begin{lstlisting}
#define IO_BLOCKS            	1 // Registers are contiguous: one block needed
#define PIN_CTRL_BASE        	((void*)0x20200000) // Pin control base address
#define PIN_CTRL_SIZE        	24 // 6 regs * 4 bytes each
#define __IO_STATE_TOTAL_SIZE	24 // Total size in bytes

static const void* bcm2835_io_addrs[IO_BLOCKS] = { PIN_CTRL_BASE };
static const unsigned bcm2835_io_sizes[IO_BLOCKS] = { PIN_CTRL_SIZE };

static const io_conf_t phys_io_conf = {
	.addrs = bcm2835_io_addrs,
	.sizes = bcm2835_io_sizes,
	.blocks = IO_BLOCKS,
	.size = __IO_STATE_TOTAL_SIZE
};
\end{lstlisting}
Of course our implementation is only a prototype, and many other registers may be included into the I/O configuration (\eg event detect, edge detect, I2C control registers, etc.).

The next step is to define the trusted behaviour of the I/O configuration.
In particular, we have pin multiplexing (\emph{pinmux}) and pin configuration (\emph{pinconf}) registers, and we assumed the following behaviour:
\begin{itemize}
	\item \itemname{pinmux registers}: they are initialised at boot time, and never change during run-time;
	\item \itemname{pinconf registers}: they are initialised at boot time, but can be modified at any time by the PLC runtime, maintaining the following invariant:
		every pin configuration (input or output mode) must be conforming to the running PLC logic.
\end{itemize}

After I/O configuration has been defined, we can describe the interface and implementation of the I/O monitor.
The abstract part is essentially made of the monitor loop, which executes the mainstrategy,
and the detection handler, called by the implementation when an I/O modification has been detected. Both codes are shown in \myalg{alg:io_iface},
and they are independent from any specific architecture or SoC model.
\algnewcommand\algorithmicforeach{\textbf{for each}}
\algdef{S}[FOR]{ForEach}[1]{\algorithmicforeach\ #1\ \algorithmicdo}
\begin{algorithm}[h]
\caption{I/O monitor interface: main loop and detection handler}
\label{alg:io_iface}
\begin{algorithmic}[1]
\Function{IOMainLoop}{\null}
	\State $t \gets$ monitor scan interval \Comment{The monitor interval in $\SI{}{ms}$}
	\State $C \gets$ physical I/O configuration \Comment{The defined I/O configuration}
	\State $T \gets$ \Call{GetIOState}{$C$} \Comment{Read trusted state from I/O registers}
	\Loop
		\ForEach{block $B \in C$ having index $i$}
			\State \Call{CheckIOState}{$B, T[i]$} \Comment{Compare current and trusted state of block $B$}
		\EndFor
		\If{monitor should stop}
			\State \Return \Comment{Return if Ghostbuster is being stopped}
		\EndIf
		\State \Call{msleep}{$t$} \Comment{Wait for next cycle}
	\EndLoop
\EndFunction
\Statex
\end{algorithmic}

\begin{algorithmic}[1]
\Function{HandleIODetection}{$D$} \Comment{React to the detection $D$}
	\State report about D
	\If{I/O debug enabled}
		\State dump entire I/O state
	\EndIf
	\If{\Call{IsLegitimate}{$D$}}
		\State \Call{UpdateIOState}{$D$} \Comment{Accept new configuration, update trusted state}
	\Else
		\State report about Pin Control Attack
		\If{I/O monitor is active}
			\State \Call{RestoreIOState}{$D$} \Comment{Reject new configuration, restore trusted state back}
		\EndIf
	\EndIf
\EndFunction
\end{algorithmic}
\end{algorithm}

Each iteration of the main loop is executed every $t~\SI{}{ms}$, where the specific value of $t$ is defined into the \verb|IO_MONITOR_INTERVAL| flag.
The loop terminates only if an external signal indicates that Ghostbuster is being stopped (see \mysec{sec:def_usage}).
The verification of the current I/O configuration is performed at block level against the golden reference, which is obtained by \textproc{GetIOState} before starting the loop.
Since we assume that the system is in a safe state when our module is deployed, the initial I/O state read from configuration registers is considered trusted.
When a modification is detected by \textproc{CheckIOState}, the detection handler is called back and the monitor strategy is applied.
Note that the handler may be called many times from the same call to \textproc{CheckIOState}, because there may be more than one single modification within one contiguous block.
The detection granularity is decided by the implementation part (\eg for each single pin in our implementation).
If the modification is considered legitimate, then the reference I/O state is updated with the new configuration.
Otherwise, we report the attack to the system and, since we have an exact detection condition which cannot give false positives,
we may safely restore the previous configuration back if the I/O monitor is in active mode.

All the low-level I/O functions are defined into the implementation part, and may be different for each target system.
These functions deal with the effective accesses to I/O registers, which depend on many factors. The implementation knows the way to access these registers,
their size and their bits arrangement. In \myalg{alg:io_impl} we show our implementation related to the BCM2835 SoC, where all registers are 32-bits wide.
\begin{algorithm}[h]
\caption{I/O monitor implementation functions}
\label{alg:io_impl}
\begin{algorithmic}[1]
\Function{GetIOState}{$C$} \Comment{Read current I/O state}
	\State $T \gets \emptyset$
	\ForEach{block $B$ in $C$}
		\ForEach{register $R \in B$}
			\State $T \gets T \cup \Call{ioread32}{$R$}$ \Comment{Save value of register $R$ into the trusted state}
		\EndFor
	\EndFor
	\State \Return $T$
\EndFunction
\Statex
\end{algorithmic}

\begin{algorithmic}[1]
\Function{CheckIOState}{$B, T_B$} \Comment{Verify block $B$ against its trusted state $T_B$}
	\ForEach{register $R \in B$ having index $i$}
		\State $C_R \gets \Call{ioread32}{$R$}$ \Comment{Read current value of register $R$}
		\State $T_R \gets T_B[i]$ \Comment{Trusted value of register $R$}
		\ForEach{pin $p \in R$ having index $j$}
			\State $d \gets C_R[j] \oplus T_R[j]$ \Comment{Difference between current and trusted value}
			\If{$d \neq 0$}
				\State $D \gets \{C_R, T_R, j, d\}$ \Comment{Fill in detection information}
				\State \Call{HandleIODetection}{$D$} \Comment{Call detection handler}
			\EndIf
		\EndFor
	\EndFor
\EndFunction
\Statex
\end{algorithmic}

\begin{algorithmic}[1]
\Function{IsLegitimate}{$D$}
	\If{$D$ is pin multiplexing}
		\State \Return false \Comment{Pin multiplexing is not allowed at run-time}
	\EndIf
	\If{$D$ is pin configuration}
		\If{new I/O configuration is conforming to PLC logic}
			\State \Return true \Comment{Change performed by the PLC runtime}
		\Else
			\State \Return false \Comment{Malicious change}
		\EndIf
	\EndIf
\EndFunction
\Statex
\end{algorithmic}

\begin{algorithmic}[1]
\Function{UpdateIOState}{$D$}
	\State $D.T_R[j] = D.C_R[j]$ \Comment{Update the trusted value according to new I/O configuration}
\EndFunction
\Statex
\end{algorithmic}

\begin{algorithmic}[1]
\Function{RestoreIOState}{$D$}
	\State \Call{iowrite32}{$D.T_R \oplus D.d$} \Comment{Restore I/O configuration to the trusted value}
\EndFunction
\end{algorithmic}
\end{algorithm}

The detection events are managed at pin level, \ie if an attacker modifies the configuration of more than one pin at a time, only one pin at a time is verified.
The critical function of the implementation part is the \textproc{IsLegitimate} function, which decides whether a configuration change is legal or not.
Based on our behavioural model, pin multiplexing can never be legitimate during run-time, so the implementation is straightforward.
In the case of pin configuration, instead, the implementation needs to check if the I/O configuration after the change is in conflict with the PLC logic.
A conflict occurs in one of the following two cases:
\begin{itemize}
	\item \itemname{write}: the pin is set as input and the logic is trying to write from it;
	\item \itemname{read}: the pin is set as output and the logic is trying to read from it.
\end{itemize}
The most challenging problem here is to figure out which operation the logic is performing on a given I/O pin.
To solve this problem, we applied the reverse engineering approach proposed in \mysec{sec:io_design}, reported below. 

Inside the BCM2835, a specific set of 32-bit registers is used to interact with I/O pins: LVL registers to read the pin value,
CLR registers to write a $0$ and SET registers to write a $1$ \cite{bcm2835}. Every register contains a bit for each pin, for a total of 32 pins per register.
To have access to the operations performed by the PLC logic on these registers, we leveraged the debug subsystem.
In ARM architecture, the debug subsystem provides two different types of debug registers: breakpoint and watchpoint (see \mysec{sec:dr_impl}).
When a pin configuration change is detected, the I/O monitor inserts a watchpoint to the corresponding LVL, CLR or SET register of the affected pin,
in order to intercept the I/O operation performed by the PLC logic. The watchpoint can be set for either a read or a write, according to the specific case we want to verify.
From a reverse engineering analysis of the PLC runtime, we found that read and write operations are performed with the following instructions, respectively:
\begin{itemize}
	\item \verb|STR R2, [R3]| (Opcode \verb|0x002083E5|, \verb|R3| contains the address of a STR or CLR register);
	\item \verb|LDR R2, [R3]| (Opcode \verb|0x002093E5|, \verb|R3| contains the address of a LVL register).
\end{itemize}
\verb|R3| always contains the virtual address targeted by our watchpoint. When a watchpoint is hit, the debug exception handler is called,
and all the execution context of the process is passed as argument. Inside the debug handler, we proceed as follows, according to the case:
\begin{itemize}
	\item \itemname{write}: we look at the content of \verb|R2| to check if the PLC logic is trying to write to the specific pin changed to input.
		If \verb|R2| contains a $1$ corresponding to the target pin, then we can conclude that the new configuration is in conflict with the logic.
	\item \itemname{read}: in this case more reverse engineering of the PLC runtime is needed to know which pins (\ie bits) are actually used by the logic as input,
		because the instruction always loads the entire register (32 pins).
		To simplify the detection in our prototype version, we supposed that the PLC logic may only have one input pin for each register.
		In this case, reading from the register implies that the PLC logic is trying to read from the given pin. Thus, we can report a conflict.
		To remove this assumption, which limits the applicability of our defense, more knowledge about the PLC runtime is required.
		As previously discussed, this knowledge can be obtained either from a deeper reverse engineering analysis or from a collaboration with the PLC runtime vendor.
		For instance, each PLC logic may be designed to contain a constant bit mask having one bit for each pin,
		where each bit specifies whether the corresponding pin is currently used as input or output. If such data was available,
		the I/O monitor could look at it instead of inserting a watchpoint and looking for the actual operations.
		This mechanism raise the bar for the attacker, who would need to alter the PLC logic code as well to conduct the attack,
		thus defeating its stealthiness.
\end{itemize}

Note that from a performance perspective our approach is feasible, because it inserts a debug exception only in case of an I/O modification to determine if it is malicious.
Thus, this may cause an overhead when a good I/O configuration is updated by the PLC runtime as well. See \mychap{chap:results} for more details about our results.


\subsubsection{Wago PLC version}

To port the I/O monitor to the Wago PLC system, only the following changes are necessary:
\begin{itemize}
	\item \itemname{I/O configuration}: the I/O configuration and the behavioural model must be redefined according to the registers used by the Wago PLC.
		In particular, the implementation may include pin configuration and pin multiplexing registers as well, plus SPI, DMA and IRQ registers.
		The way to access registers is pretty much equal to the Raspberry Pi system, because are both ARM architectures with 32-bit wide registers.
		On Wago PLC, pin configuration and pin multiplexing are managed by two different set of registers. Therefore, the low-level implementation
		is even simplified, because it does not need to distinguish configuration and multiplexing bits inside registers.
		However, including other registers (\eg SPI, DMA, IRQ, etc.) may lead to more complex behavioural models which need to be analysed.
	\item \itemname{Debug subsystem}: given the engineering problem described later in \mysec{sec:dr_impl}, the same solution using watchpoint cannot be directly applied.
		To get the needed information about the PLC logic, two ways are possible. Either the debug interface is implemented into the kernel for AM3517 SoC,
		or a better integration with the PLC runtime is required, as already discussed.
\end{itemize}


\subsection{DR monitor}
\label{sec:dr_impl}

The DR monitor aims to protect the debug subsystem. As described for the design phase, the DR monitor needs to accomplish two goals:
\begin{itemize}
	\item disable debug interface access from userspace;
	\item watch over debug registers value to detect malicious events.
\end{itemize}

In Linux, the debug user interface is based on the following functions \cite{hw-breakpoint}:
\begin{lstlisting}
struct perf_event* register_user_hw_breakpoint(struct perf_event_attr* attr,
                                               perf_overflow_handler_t handler,
                                               struct task_struct* tsk);

int modify_user_hw_breakpoint(struct perf_event *bp,
                              struct perf_event_attr *attr);

void unregister_hw_breakpoint(struct perf_event *bp);
\end{lstlisting}
To disable these funtions, we dynamically patched the kernel text replacing the prologue instructions of each function.
In particular, they have the following prologue:
\begin{Verbatim}[fontsize=\small]
	c00d3cb0 <register_user_hw_breakpoint>:
	c00d3cb0: e1a0c00d  mov  ip, sp
	c00d3cb4: e92dd800  push {fp, ip, lr, pc}
	c00d3cb8: e24cb004  sub  fp, ip, #4
	c00d3cbc: e24dd008  sub  sp, sp, #8
	[...]

	c00d3e30 <modify_user_hw_breakpoint>:                                
	c00d3e30: e1a0c00d  mov  ip, sp
	c00d3e34: e92ddff0  push {r4, r5, r6, r7, r8, r9, sl, fp, ip, lr, pc}
	c00d3e38: e24cb004  sub  fp, ip, #4
	c00d3e3c: e24dd00c  sub  sp, sp, #12
	[...]

	c00d3ce8 <unregister_hw_breakpoint>:
	c00d3ce8: e1a0c00d  mov   ip, sp
	c00d3cec: e92dd800  push  {fp, ip, lr, pc}
	c00d3cf0: e24cb004  sub   fp, ip, #4
	[...]
\end{Verbatim}
We replaced the prologue instructions with the following ones:
\begin{enumerate}
	\item \itemname{return value}: if the function is not void (\eg in the first two cases), we need to return a value to the caller.
		Thus, we move the value \verb|-EACCES| ($-13$) into R0 by using the move-negate instruction: \verb|mvn r0, #0xC|.
		\verb|EACCES| is defined into the \verb|errno-base.h| kernel header as \verb|Permission denied|;
	\item \itemname{branch}: jump back to the caller by branching to the link register: \verb|bx lr|.
\end{enumerate}
After the patch, the kernel text becomes the following:
\begin{Verbatim}[fontsize=\small]
	c00d3cb0 <register_user_hw_breakpoint>:
	c00d3cb0: e3e0000c  mvn  r0, #0xC
	c00d3cb4: e12fff1e  bx   lr
	[...]

	c00d3e30 <modify_user_hw_breakpoint>:                                
	c00d3e30: e3e0000c  mvn  r0, #0xC
	c00d3e34: e12fff1e  bx   lr
	[...]

	c00d3ce8 <unregister_hw_breakpoint>:
	c00d3ce8: e12fff1e  bx   lr
	[...]
\end{Verbatim}

Once the user access to debug registers has been disabled, we activate the DR monitor to protect them from attackers who gain kernel access.
As for the I/O monitor, we report the interface part and the implementation.
The abstract strategy of the DR monitor is rather simple, and is listed in \myalg{alg:dr_iface}.
The scheme is a simplification of the I/O monitor, in which any debug register modification is considered malicious.
Since in our implementation the I/O monitor makes use of the debug subsystem, we used the DR monitor to mediate any access to debug registers.
Therefore, any modification performed by the I/O monitor is excluded from the detection, and the DR state is protected from concurrent access by a mutual exclusion mechanism.
\begin{algorithm}[h]
\caption{DR monitor interface: main loop and detection handler}
\label{alg:dr_iface}
\begin{algorithmic}[1]
\Function{DRMainLoop}{\null}
	\State $m_T \gets$ mutex \Comment{Global mutex to protect trusted state}
	\State $t \gets$ monitor scan interval \Comment{The monitor interval in $\SI{}{ms}$}
	\State $T \gets$ \Call{GetDRState}{\null} \Comment{Read trusted state from debug registers}
	\Loop
		\State \Call{lock}{$m_T$}
		\State \Call{CheckDRState}{$T$} \Comment{Compare current and trusted state of debug registers}
		\State \Call{unlock}{$m_T$}
		\If{monitor should stop}
			\State \Return \Comment{Return if Ghostbuster is being stopped}
		\EndIf
		\State \Call{msleep}{$t$} \Comment{Wait for next cycle}
	\EndLoop
\EndFunction
%\Statex
\end{algorithmic}
\end{algorithm}
\begin{algorithm}[h]
\ContinuedFloat
\begin{algorithmic}[1]
\Function{HandleDRDetection}{$D$} \Comment{React to the detection $D$}
	\If{DR debug enabled}
		\State dump entire DR state
	\EndIf
	\State report about Pin Control Attack
	\If{DR monitor is active}
		\State \Call{RestoreDRState}{$D$} \Comment{Restore debug register trusted state}
	\EndIf
\EndFunction
\end{algorithmic}
\end{algorithm}
The interface exposed to the I/O monitor is listed in \myalg{alg:dr_ioiface}.
\begin{algorithm}[h]
\caption{DR monitor interface for I/O monitor}
\label{alg:dr_ioiface}
\begin{algorithmic}[1]
\Function{SetDR}{$r$} \Comment{Set a debug register $r$}
	\State \Call{lock}{$m_T$}
	\State \Call{register\_wide\_hw\_breakpoint}{$r$} \Comment{Use kernel interface for debug registers}
	\State $T \gets$ \Call{GetDRState}{\null} \Comment{Update trusted state}
	\State \Call{unlock}{$m_T$}
\EndFunction
\Statex
\end{algorithmic}

\begin{algorithmic}[1]
\Function{ResetDR}{$r$} \Comment{Reset a previously set debug register $r$}
	\State \Call{lock}{$m_T$}
	\State \Call{unregister\_wide\_hw\_breakpoint}{$r$} \Comment{Use kernel interface for debug registers}
	\State $T \gets$ \Call{GetDRState}{\null} \Comment{Update trusted state}
	\State \Call{unlock}{$m_T$}
\EndFunction
\end{algorithmic}
\end{algorithm}
%the interface is usable by the attacker?

All the details about debug registers are handled by the implementation part, shown in \myalg{alg:dr_impl}.
\begin{algorithm}[h]
\caption{DR monitor implementation functions}
\label{alg:dr_impl}
\begin{algorithmic}[1]
\Function{GetDRState}{\null} \Comment{Read current I/O state}
	\State $T \gets \emptyset$
	\State $D \gets hardware debug registers$
	\ForEach{debug register $r \in D$}
		\State $T \gets T \cup \Call{DRread}{$r$}$ \Comment{Save value of debug register $r$ into $T$}
	\EndFor
	\State \Return $T$
\EndFunction
\Statex
\end{algorithmic}

\begin{algorithmic}[1]
\Function{CheckDRState}{$T$} \Comment{Verify debug registers against trusted state $T_B$}
	\ForEach{debug register $r \in B$ having index $i$}
		\State $C_r \gets \Call{DRread}{$r$}$ \Comment{Read current value of register $r$}
		\State $T_r \gets T[i]$ \Comment{Trusted value of register $r$}
		\State $d \gets C_r \oplus T_r$ \Comment{Difference between current and trusted value}
		\If{$d \neq 0$}
			\State $D \gets \{C_r, T_r, r\}$ \Comment{Fill in detection information}
			\State \Call{HandleDRDetection}{$D$} \Comment{Call detection handler}
		\EndIf
	\EndFor
\EndFunction
%\Statex
\end{algorithmic}
\end{algorithm}
\begin{algorithm}[h]
\ContinuedFloat
\begin{algorithmic}[1]
\Function{RestoreDRState}{$D$}
	\State \Call{DRwrite}{$D.r, D.T_r$} \Comment{Restore debug register to the trusted value}
\EndFunction
\end{algorithmic}
\end{algorithm}
The ARM architecture supports two type of hardware debug registers: breakpoints for instruction addresses, and watchpoints for data addresses.
The DR monitor protects both of them, and the detection is handled at register level. Each register is accessed through specific coprocessor instructions.
A special read-only register, the \emph{Debug ID Register} (DIDR), specifies the number of debug registers available and other useful information on the SoC configuration.
In particular, since the BCM2835 SoC supports 2 watchpoints and 6 breakpoints, the DR monitor can be deployed.


\subsubsection{Wago PLC version}

To adapt the DR monitor for Wago PLC, the following engineering problem needs to be solved.
The AM3517 SoC model embedded into this PLC provides a different interface for debug registers. In particular, it uses a memory mapped interface
instead of specific coprocessor instructions \cite{am35x}.
Unfortunately, it turns out that this particular interface is not supported by the Linux kernel debug framework, as reported in \cite{dr-mapped}:
``The memory-mapped extended debug interface is unsupported due to its unreliability in real implementations''.
Apart from this technical problem, the DR monitor is designed to be applicable to any other ARM implementation that supports debug registers.
To port it for other architectures than ARM, only the implementation part must be adapted.


\subsection{MAP monitor}

TODO MAP monitor.


\subsubsection{Wago PLC version}

TODO Wago PLC version.


\subsection{Module usage}
\label{sec:def_usage}

%Our defense can be either dynamically inserted as a Loadable Kernel Module (LKM), or can be built-in into the Linux kernel.
%The second approach requires a kernel re-compilation to obtain a new kernel image including our module.
%The security implications of these two approaches are discussed in \mysec{sec:def_sec}.
TODO Module usage.


\section{Security considerations}
\label{sec:def_sec}

TODO Security considerations.


\chapter{Experimental Results}
\label{chap:results}

TODO Experimental Results.


\chapter{Conclusions}
\label{chap:conclusions}

This final chapter concludes the presented work, highlighting its main contribution and drawbacks, and suggesting possible future works.


\section{Contribution and drawbacks}
\label{sec:contrib}

The scientific contribution of this work is made of two main components:
\begin{enumerate}
	\item analyse I/O attack and prove that it is actually feasible on real PLCs;
	\item design and implement a possible countermeasure, achieving a good detection rate without causing an unacceptable performance overhead.
\end{enumerate}
Both parts can obviously be improved and continued, as discussed later in \mysec{sec:future}.

By analysing the attack, we demonstrated that it is possible to tamper with the I/O operations even in a real PLC architecture, where input/output is managed by external modules.
Furthermore, we were able to clearly define the requirements and the attack vectors available for the attacker, useful to abstract the problem and design a proper solution.
In the first part of \mychap{chap:defense}, we tried to define abstract strategies that could be applied to any embedded system to tackle any kind of I/O attack.
Furthermore, the implementation has been designed to minimise the effort required to adapt the solution to different systems and architectures.

Nevertheless, our detection system is still the first step against I/O attack, and, of course, it has some limitations.
First, it is not a complete solution, in the sense that I/O attacks could still be possible if the attacker is able to gain accurate timing precision and evade our time-based monitors.
Of course this raises the bar for the attacker, who has to conduct more elaborated attacks to achieve the same result.
Moreover, our detection monitors can be improved and optimized to increase their detection rate, factually discouraging any malicious attempt.
Second, an attacker who gains kernel level access may still be able to circumvent our defensive mechanism, and other techniques should be used in combination with it,
as previously discussed in \mysec{sec:def-sec}. In the next section we discuss some possible future works regarding both our contributions.


\section{Future works}
\label{sec:future}

During each phase of this work, many ideas about possible future works came out.
First, for the attack phase, further possibilities can be analysed to implement more elaborated attacks on the real PLC with external I/O modules.
For instance, in our Wago PLC, it may be possible to fake both input and output by leveraging SPI bitbanging technique.
A possible implementation of this attack can do the following:
\begin{itemize}
	\item synchronise itself with PLC I/O operation;
	\item disable next PLC I/O operation using one of the techniques presented in \mysec{sec:attack-plc};
	\item perform malicious I/O operation within the next scan cycle, \eg shifted by $\SI{5}{ms}$ after the real one (accurate synchronisation is required).
\end{itemize}
Note that all these operations can be done with the same assumptions made for the described I/O attack, without hooking any function nor modifying PLC logic/runtime.
Moreover, several vendors are currently producing different PLCs having a small subset of I/O interfaces directly available without the need of external modules.
If this feature, known as \emph{Integrated I/O}, is actively used on a control system, then the attacker can have direct access to sensors and actuators as well,
without needing to go through the additional level of indirection caused by external modules. In this case, the original version of Pin Control Attack would be applicable.

For the defense phase, many improvements and extension are doable.
First, detection rates of DR and I/O monitor can be improved by synchronising them with respect to the timing of the PLC scan cycle.
In fact, if they are triggered in proximity of each PLC I/O operation, detecting more sophisticated attacks becomes easier.
To get accurate results, this approach needs to be tested in a real-time system.
Of course, if the vendor decides to integrate the defense with the PLC runtime itself, this would be the optimal solution, because it can simply check I/O configuration
right before performing an I/O operation.

Another benefit deriving from an integration with the PLC runtime would be a significant simplification of our defense,
in particular when it has to check whether the I/O configuration is in line with the current PLC logic.
We largely discussed this aspect in \mysec{sec:io-design}. Briefly, if the PLC runtime is designed to be aware of the defense,
it can provide a simpler interface for Ghostbuster to check if a conflict between configuration and logic occurred.
In fact, in our prototype implementation, described in \mysec{sec:io-impl}, we leveraged reverse engineering techniques to collect the necessary information.

Additionally, our solution may be extended with a performance monitor, to cover a larger set of attacks and increase the overall detection rate.
We can argue that a performance-based mechanism would be effective on a PLC system, which executes the same operations at each scan cycle, thus, it is very stable.
As previously discussed in \mysec{sec:pre-analysis}, the operations performed by the PLC runtime when a new logic is uploaded should be excluded from the detection.
Again, an integration with the PLC runtime would be helpful to design this behaviour as well.
For instance, the performance monitor may be disabled and re-enabled by the PLC runtime process through authenticated commands sent to the defense module.
All these proposed approaches should be evaluated with respect to the overhead imposed to the system.

Finally, our entire solution can be deployed within a Trusted Execution Environment, such as ARM TrustZone \cite{trustzone}.
However, this approach may not be feasible due to the excessive CPU performance degradation to switch between secure and non-secure world.
Therefore, a carefully designed solution is required to minimise its overhead.



\begin{thebibliography}{99}

\bibitem{ghostplc}
A.Abbasi, M.Hashemi, E.Zambon, S.Etalle,
``Stealth Low-Level Manipulation of Programmable Logic Controllers I/O By Pin Control Exploitation'',
TODO Black Hat Conference 2016.



\end{thebibliography}



%\begin{appendices}
%\setcounter{table}{0}
%\setcounter{figure}{0}
%\renewcommand\thetable{\thesection\arabic{table}}
%\renewcommand\thefigure{\thesection\arabic{figure}}

%\end{appendices}


\end{document}
