\chapter{Conclusions}
\label{chap:conclusions}

This final chapter concludes the presented work, highlighting its main contribution and drawbacks, and suggesting possible future works.


\section{Contribution and drawbacks}
\label{sec:contrib}

The scientific contribution of this work is made of two main components:
\begin{enumerate}
	\item analyse I/O attack and prove that it is actually feasible on real PLCs;
	\item design and implement a possible countermeasure, achieving a good detection rate without causing an unacceptable performance overhead.
\end{enumerate}
Both parts can obviously be improved and continued, as discussed later in \mysec{sec:future}.

By analysing the attack, we demonstrated that it is possible to tamper with the I/O operations even in a real PLC architecture, where input/output is managed by external modules.
Furthermore, we were able to clearly define the requirements and the attack vectors available for the attacker, useful to abstract the problem and design a proper solution.
In the first part of \mychap{chap:defense}, we tried to define abstract strategies that could be applied to any embedded system to tackle any kind of I/O attack.
Furthermore, the implementation has been designed to minimise the effort required to adapt the solution to different systems and architectures.

Nevertheless, our detection system is still the first step against I/O attack, and, of course, it has some limitations.
First, it is not a complete solution, in the sense that I/O attacks could still be possible if the attacker is able to gain accurate timing precision and evade our time-based monitors.
Of course this raises the bar for the attacker, who has to conduct more elaborated attacks to achieve the same result.
Moreover, our detection monitors can be improved and optimized to increase their detection rate, factually discouraging any malicious attempt.
Second, an attacker who gains kernel level access may still be able to circumvent our defensive mechanism, and other techniques should be used in combination with it,
as previously discussed in \mysec{sec:def-sec}. In the next section we discuss some possible future works regarding both our contributions.


\section{Future works}
\label{sec:future}

During each phase of this work, many ideas about possible future works came out.
First, for the attack phase, further possibilities can be analysed to implement more elaborated attacks on the real PLC with external I/O modules.
For instance, in our Wago PLC, it may be possible to fake both input and output by leveraging SPI bitbanging technique.
A possible implementation of this attack can do the following:
\begin{itemize}
	\item synchronise itself with PLC I/O operation;
	\item disable next PLC I/O operation using one of the techniques presented in \mysec{sec:attack-plc};
	\item perform malicious I/O operation within the next scan cycle, \eg shifted by $\SI{5}{ms}$ after the real one (accurate synchronisation is required).
\end{itemize}
Note that all these operations can be done with the same assumptions made for the described I/O attack, without hooking any function nor modifying PLC logic/runtime.
Moreover, several vendors are currently producing different PLCs having a small subset of I/O interfaces directly available without the need of external modules.
If this feature, known as \emph{Integrated I/O}, is actively used on a control system, then the attacker can have direct access to sensors and actuators as well,
without needing to go through the additional level of indirection caused by external modules. In this case, the original version of Pin Control Attack would be applicable.

For the defense phase, many improvements and extension are doable.
First, detection rates of DR and I/O monitor can be improved by synchronising them with respect to the timing of the PLC scan cycle.
In fact, if they are triggered in proximity of each PLC I/O operation, detecting more sophisticated attacks becomes easier.
To get accurate results, this approach needs to be tested in a real-time system.
Of course, if the vendor decides to integrate the defense with the PLC runtime itself, this would be the optimal solution, because it can simply check I/O configuration
right before performing an I/O operation.

Another benefit deriving from an integration with the PLC runtime would be a significant simplification of our defense,
in particular when it has to check whether the I/O configuration is in line with the current PLC logic.
We largely discussed this aspect in \mysec{sec:io-design}. Briefly, if the PLC runtime is designed to be aware of the defense,
it can provide a simpler interface for Ghostbuster to check if a conflict between configuration and logic occurred.
In fact, in our prototype implementation, described in \mysec{sec:io-impl}, we leveraged reverse engineering techniques to collect the necessary information.

Additionally, our solution may be extended with a performance monitor, to cover a larger set of attacks and increase the overall detection rate.
We can argue that a performance-based mechanism would be effective on a PLC system, which executes the same operations at each scan cycle, thus, it is very stable.
As previously discussed in \mysec{sec:pre-analysis}, the operations performed by the PLC runtime when a new logic is uploaded should be excluded from the detection.
Again, an integration with the PLC runtime would be helpful to design this behaviour as well.
For instance, the performance monitor may be disabled and re-enabled by the PLC runtime process through authenticated commands sent to the defense module.
All these proposed approaches should be evaluated with respect to the overhead imposed to the system.

Finally, our entire solution can be deployed within a Trusted Execution Environment, such as ARM TrustZone \cite{trustzone}.
However, this approach may not be feasible due to the excessive CPU performance degradation to switch between secure and non-secure world.
Therefore, a carefully designed solution is required to minimise its overhead.
