\chapter{Related work}
\label{chap:related}

In this chapter we summarise the state of the art about security of embedded systems at the time of writing.
First, we discuss about the attacks of the recent years, showing how the embedded systems security concerns are rising.
Next, we analyse the defense mechanisms currently available in the literature, realising that they are still in a very early stage of their life.

\section{Attacks}

In the past few years a lot of attacks targeted embedded systems: most notably the infamous Stuxnet \cite{stuxnet} targeting an Iranian nuclear facility in 2010.
More recently Grandgenett et al. \cite{io-command} analysed the authentication protocol between the RSLogix 5000 software and the PLC, based on a simple challenge-response mechanism.
Since the protocol lacks freshness in its messages, is vulnerable to replay attacks, through which an attacker could repeat privileged commands to the PLC.
Furthermore, they found that both the decoding of the challenge and the encoding of the response use an RSA-2048 key which is hard-coded in the RSLogix software,
and it is actually valid for any Rockwell/Allen Bradley PLC.
This indicates how the security mechanisms of these systems often have a really poor design, if any.

Papp et al. \cite{taxonomy} analysed the existing attacks on embedded systems, relying on the proceedings of security conferences, with a focus on computer hacking,
and on the Internet for media reports, blogs and mailing list. They built a taxonomy based on five dimensions: precondition, vulnerability, target, attack method and effect of the attack.

For our purpose, we may classify the attacks found in literature using a simpler criterion based on the attack method. We may distinguish three main categories:
\begin{enumerate}
	\item \itemname{Firmware modification}: all the attacks aiming to upload a malicious firmware version (or part of it) in the device belong to this category.
	\item \itemname{Logic modification}: this category consists of the attacks that modify the PLC logic in some way. In this case the integrity of the firmware is not violated,
		but a malicious program, or logic, is inserted into the PLC to make it misbehave during the control process.
	\item \itemname{Control flow modification}: it includes the attacks that alter the normal control flow of a process by leveraging classic programming vulnerabilities
		such as buffer overflow or expired pointer dereference.
\end{enumerate}

We briefly report about these different kind of attacks in the following sections.


\subsection{Firmware modification attacks}

Almost all modern embedded systems provide a way to update the firmware, and the attackers could exploit this feature to upload its own malicious firmware.
Basnight et al \cite{firmware-mod} reverse engineered an Allen-Bradley ControlLogix L61 PLC firmware showing how to bypass the
firmware update validation method and successfully upload a counterfeit firmware.
Peck et al. \cite{ethernet-vuln} demonstrate how using commonly available software an attacker can write and load his malicious firmware into Ethernet cards of devices
used in control systems, potentially infecting the entire industrial control system.
Cui et al. \cite{print-vuln} discovered a vulnerability in the HP-RFU (Remote Firmware Update) feature of LaserJet printers,
that allows remote attackers to make persistent modifications to the printer's firmware by simply printing to it.


\subsection{Logic modification attacks}

Stuxnet \cite{stuxnet} belongs to this category. Along with its several components, mainly used to replicate and control the malware,
its core is essentially an infected version of a SCADA software library used to program the PLC itself.
By hooking some of the library functions it is able to load infected code and data blocks into the PLC and hide itself from the operator.
McLaughlin et al. \cite{dynamic-payload,sabot} evaluated some techniques and implemented a tool (SABOT) to infer the structure of a physical plant and craft a dynamic payload,
allowing an attacker to cause an unsafe behavior without having a deep \emph{a priori} knowledge of the target physical process.
Similar techniques might mitigate the precondition needed by an attack, making it even more viable.
Beresford \cite{siemens-s7} showed how the PLCs and the protocols used for communication in control systems were built without any security in mind,
and demonstrated that they are affected by many vulnerabilities which may also enable the attacker to know the current configuration and rewrite the PLC logic.
More recently, Klick et al. \cite{plc-network} used an internet-facing PLC as a network gateway by prepending a backdoor, made of a port scanner and a SOCKS proxy,
to the existing logic code of the PLC.


\subsection{Control flow modification}

Many recent advisories \cite{schneider-bof,rockwell-vuln,rockwell-vuln2,elcsoft-vuln} from ICS-CERT (Industrial Control System Cyber Emergency Response Team)
report about various programming vulnerabilities affecting both PLC firmwares and control softwares. Most of them allow remote code execution and could be exploited
without requiring particularly high skills.
The vulnerabilities discovered by Beresford \cite{siemens-s7} also allow the attacker to insert a payload into the logic and subvert the control flow to execute
malicious code. Nevertheless, the majority of the PLCs run the applications with root privileges, so it is quite simple for an attacker to get a root shell.
One of the most dangerous kind of control flow attacks consists of ROP (Return-Oriented Programming) techniques, or similar variants \cite{jop,no-ret}
which leverage different sequence of instructions equivalent to a return instruction.
Since code vulnerabilities may affect embedded systems \cite{schneider-bof,rockwell-vuln,rockwell-vuln2,elcsoft-vuln,siemens-s7}, ROP techniques
are applicable as well. Furthermore, due to the limitations imposed by these systems, is even more challenging to defend against them.


\section{Defenses}


