\documentclass[pdfa,cucitura]{toptesi}

\hypersetup{
    pdfpagemode={UseOutlines},
    bookmarksopen,
    pdfstartview={FitH},
    colorlinks,
    linkcolor={blue},
    citecolor={red},
    urlcolor={blue}
}

%
% Phantom space for abbreviations
%
\usepackage{xspace}
%
% To insert doi identifiers
%
\usepackage{doi}
\renewcommand{\doitext}{DOI }
%
% Improve citations from biblio
%
\usepackage{cite}
%
% This is to create hyperlinks for index, URLs and citations
% (now we can use the command \url{...} to create URL with hyperlink)
% 
%\usepackage{color}
%\usepackage[a4paper,colorlinks=true,urlcolor=blue,citecolor=blue,linkcolor=blue,bookmarks=false]{hyperref}
%
% This allows inclusion of pictures.
% Create figures with PowerPoint and then export them individually
% in PDF, PNG, JPEG, or GIF format (in order of preference)
%
%\usepackage[pdftex]{graphicx}
%\DeclareGraphicsExtensions{.pdf,.png,.jpg,.gif}
%
% For pasting text files
%
%\usepackage{fancyvrb}
%
% Used to express formulas like n^th
%
%\usepackage{mathtools}
%
% For table formatting
%
%\usepackage{longtable}
%\usepackage{makecell}
%\usepackage{multirow}
%
% For plotting results
%
%\usepackage{pgfplots}
%\pgfplotsset{compat=newest}
%
% For placing floats
%
%\usepackage{placeins}
%
% For Appendix section
%
%\usepackage[titletoc,toc,title]{appendix}
%
% Definition of margins
%
%\usepackage[top=2cm,bottom=2cm,left=2cm,right=2cm]{geometry}
%
% Paragraph skip and indent
%
%\setlength\parskip{\medskipamount}
%\setlength\parindent{0pt}
%
% For itemize and enumerate spacing
%
\usepackage{enumitem}
% 
% For International System of Units (SI)
%
%\usepackage[binary-units]{siunitx}
%\sisetup{per-mode=symbol} % 1/s instead of s^-1
%
% For programming code
%
\usepackage{listings}
\lstset{
basicstyle=\ttfamily,
columns=fullflexible,
xleftmargin=3ex,
breaklines,
breakatwhitespace,
escapechar=`
}

% Some page parameters
\setlength{\parskip}{\medskipamount}
% Horizontal rule
\newcommand{\HRule}{\rule{\linewidth}{0.2mm}}


%
% Frequently used abbreviations.
% - example: \ie this is an example
%
\def\eg{e.g.\xspace}
\def\ie{i.e.\xspace}
\def\chap{Chapter\xspace}
\def\sec{Section\xspace}
\def\myfig#1{Fig.~#1\xspace} % usage: \myfig{\ref{fig:tag}}
\def\mytab#1{Tab.~#1\xspace}
\newcommand{\ltx}{\LaTeX\xspace}
\newcommand{\txw}{TeXworks\xspace}
\newcommand{\mik}{MikTex\xspace}
\newcommand{\html}{HTML\xspace}
\newcommand{\xhtml}{XHTML\xspace}
\newcommand{\cmd}[1]{\texttt{#1}\xspace}

% Styles
\newcommand{\itemname}{\textbf}
\newcommand{\thead}{\textbf}

% To cite RFC, es. \rfc{822}
\newcommand{\rfc}[1]{RFC-#1\xspace}
% To cite file (es. \file{autoexec.bat}) or fake URI (i.e. \file{http://www.lioy.it/})
% for real URIs use \url o \href
\newcommand{\file}[1]{\texttt{#1}\xspace}
% For inline code
\newcommand{\code}[1]{\lstinline|#1|}
% Backslash
\newcommand{\bs}{\textbackslash}
% Term definition with insertion into the index
\newcommand{\tdef}[1]{\textit{#1}\index{#1}}
% Meta-term
\newcommand{\meta}[1]{\textit{#1}}


\begin{document}
\selectlanguage{english}

\CorsoDiLaureaIn{Corso di Laurea Magistrale in }
\TesiDiLaurea{Master Thesis}
\AdvisorName{Supervisor}
\CandidateName{Candidate}


\logosede{res/logopolito}
\ateneo{Politecnico di Torino}

\titolo{Pin Control Protection}
\sottotitolo{Protecting Linux Kernel Pin Control Subsystem from Pin Control Attack}

\corsodilaurea{Ingegneria Informatica}

\candidato{Andrea \textsc{Genuise}}

\relatore{prof.\ Antonio Lioy}

\sedutadilaurea{\textsc{Academic~Year} 2016-2017}


\errorcontextlines=9

\frontespizio
\paginavuota
\newpage

\advance\voffset -5mm
\advance\textheight 30mm


\begin{dedica}
\textdagger\ A nonno Carmelo
\end{dedica}


\sommario

Embedded systems use Input/Output to interact with sensors and actuators, thus controlling the physical world within the context of an industrial process.
Particularly when deployed in mission critical systems, the I/O has to be both reliable and secure.
The I/O interface (made of pins) is controlled by a set of registers: by altering these registers one can change the behavior of the chip in a dramatic way.
This feature is exploitable by attackers, who can tamper with the integrity or the availability of legitimate I/O operations,
factually changing how a PLC interacts with the outside world. This novel kind of attack has been called Pin Control Attack.
In this thesis we design a possible countermeasure to such an attack, showing its effectiveness and the impact on embedded systems
which usually have limited resources.


\ringraziamenti

TODO Acknowledgements.


%\tablespagetrue
%\figurespagetrue

\indici

\mainmatter


\chapter{Introduction}
\label{chap:intro}

Embedded systems are widely used today in various applications, from consumer such as cars, cell phones, home automation, to critical infrastructures
like power plants and power grids, water, gas or electricity distribution systems and production systems for food and other products.
Within the context of an Industrial Control System (ICS), these systems are known as PLCs (Programmable Logic Controllers).

But security for these systems is an open question and could be a more difficult long-term problem than security for desktop and enterprise computing,
both for their limited capabilities and for the physical side effects a security breach could lead to, including property damage, personal injury, death and
even environmental or nuclear disaster.

The PLCs control the outside world via their I/O interfaces: therefore, they must be both reliable and secure.
Digging into their architecture, we know that the I/O interfaces of PLCs (e.g., GPIO, SCI, JTAG, etc.),
are usually controlled by a so called System on a Chip (SoC), an integrated circuit that combines multiple I/O interfaces.
In turn, the pins in a SoC are managed by a pin controller, a subsystem of SoC, through which one can configure the operating mode of the pins, such as the input or output mode.
One of the most peculiar aspects of a pin controller is that its behavior is determined by a set of registers: by altering these registers one can change the behavior
of the chip in a dramatic way. In \cite{ghostplc}, Abbasi et al. showed how this feature is exploitable by attackers, who can tamper with
the integrity or the availability of legitimate I/O operations, factually changing how a PLC interacts with the outside world.

Based on these observations, they introduced a novel attack technique against PLCs, called Pin Control Attack.
The salient features of this new class of attacks are:
\begin{enumerate}
	\item It is intrinsically stealth. The alteration of the pin configuration does not generate any interrupt, preventing the Operating System (OS) to react to it.
	\item It is entirely different in execution from traditional techniques such as manipulation of kernel structures or system call hooking, which are typically
		monitored by anti-rootkit protection systems.
	\item It is viable. It is possible to build concrete attack using it.
\end{enumerate}

To overcome this sophisticated attack, we propose a defensive mechanism to extend the existing Linux Kernel Pin Control Subsystem to be able to detect our novel attack.
It is a challenging task for two reasons:
\begin{enumerate}
	\item The Pin Control lacks hardware interrupts, so is not possible to directly react to any configuration change. More complex detection mechanisms are needed
		in order to achieve the highest possible detection rate.
	\item The resources available within an embedded system like a PLC are extremely limited. Therefore, our solution must be extremely agile and light
		since the smallest delay in the PLC I/O operation can have unintended consequences for the controlled process.
\end{enumerate}

In \chap \ref{chap:related} we show how Pin Control Attack is different from the majority of attacks in the literature, and discuss the protection mechanisms currently available
for embedded systems, showing that no one of these is actually capable of detecting the attack.
In \chap \ref{chap:design} we present the architecture of our proposed Pin Control Protection, then we describe the implemented modules and provide an user manual.
The \chap \ref{chap:results} provides the results obtained during the experiments, showing the detection rate and the performance overhead on a PLC environment.
Finally, in \chap \ref{chap:conclusions} we analyse the shortcomings of our defense and the possible future works and improvements.


\chapter{Related work}
\label{chap:related}

In this chapter we summarise the state of the art about security of embedded systems at the time of writing.
First, we discuss about the attacks of the recent years, showing how the embedded systems security concerns are rising.
Next, we analyse the defense mechanisms currently available in the literature, realising that they are still in a very early stage of their life.

\section{Attacks}

In the past few years a lot of attacks targeted embedded systems: most notably the infamous Stuxnet \cite{stuxnet} targeting an Iranian nuclear facility in 2010.
More recently Grandgenett et al. \cite{io-command} analysed the authentication protocol between the RSLogix 5000 software and the PLC, based on a simple challenge-response mechanism.
Since the protocol lacks freshness in its messages, is vulnerable to replay attacks, through which an attacker could repeat privileged commands to the PLC.
Furthermore, they found that both the decoding of the challenge and the encoding of the response use an RSA-2048 key which is hard-coded in the RSLogix software,
and it is actually valid for any Rockwell/Allen Bradley PLC.
This indicates how the security mechanisms of these systems often have a really poor design, if any.

Papp et al. \cite{taxonomy} analysed the existing attacks on embedded systems, relying on the proceedings of security conferences, with a focus on computer hacking,
and on the Internet for media reports, blogs and mailing list. They built a taxonomy based on five dimensions: precondition, vulnerability, target, attack method and effect of the attack.

For our purpose, we may classify the attacks found in literature using a simpler criterion based on the attack method. We may distinguish three main categories:
\begin{enumerate}
	\item \itemname{Firmware modification}: all the attacks aiming to upload a malicious firmware version (or part of it) in the device belong to this category.
	\item \itemname{Logic modification}: this category consists of the attacks that modify the PLC logic in some way. In this case the integrity of the firmware is not violated,
		but a malicious program, or logic, is inserted into the PLC to make it misbehave during the control process.
	\item \itemname{Control flow modification}: it includes the attacks that alter the normal control flow of a process by leveraging classic programming vulnerabilities
		such as buffer overflow or expired pointer dereference.
\end{enumerate}

We briefly report about these different kind of attacks in the following sections.


\subsection{Firmware modification attacks}

Almost all modern embedded systems provide a way to update the firmware, and the attackers could exploit this feature to upload its own malicious firmware.
Basnight et al \cite{firmware-mod} reverse engineered an Allen-Bradley ControlLogix L61 PLC firmware showing how to bypass the
firmware update validation method and successfully upload a counterfeit firmware.
Peck et al. \cite{ethernet-vuln} demonstrate how using commonly available software an attacker can write and load his malicious firmware into Ethernet cards of devices
used in control systems, potentially infecting the entire industrial control system.
Cui et al. \cite{print-vuln} discovered a vulnerability in the HP-RFU (Remote Firmware Update) feature of LaserJet printers,
that allows remote attackers to make persistent modifications to the printer's firmware by simply printing to it.


\subsection{Logic modification attacks}

Stuxnet \cite{stuxnet} belongs to this category. Along with its several components, mainly used to replicate and control the malware,
its core is essentially an infected version of a SCADA software library used to program the PLC itself.
By hooking some of the library functions it is able to load infected code and data blocks into the PLC and hide itself from the operator.
McLaughlin et al. \cite{dynamic-payload,sabot} evaluated some techniques and implemented a tool (SABOT) to infer the structure of a physical plant and craft a dynamic payload,
allowing an attacker to cause an unsafe behavior without having a deep \emph{a priori} knowledge of the target physical process.
Similar techniques might mitigate the precondition needed by an attack, making it even more viable.
Beresford \cite{siemens-s7} showed how the PLCs and the protocols used for communication in control systems were built without any security in mind,
and demonstrated that they are affected by many vulnerabilities which may also enable the attacker to know the current configuration and rewrite the PLC logic.
More recently, Klick et al. \cite{plc-network} used an internet-facing PLC as a network gateway by prepending a backdoor, made of a port scanner and a SOCKS proxy,
to the existing logic code of the PLC.


\subsection{Control flow modification}

Many recent advisories \cite{schneider-bof,rockwell-vuln,rockwell-vuln2,elcsoft-vuln} from ICS-CERT (Industrial Control System Cyber Emergency Response Team)
report about various programming vulnerabilities affecting both PLC firmwares and control softwares. Most of them allow remote code execution and could be exploited
without requiring particularly high skills.
The vulnerabilities discovered by Beresford \cite{siemens-s7} also allow the attacker to insert a payload into the logic and subvert the control flow to execute
malicious code. Nevertheless, the majority of the PLCs run the applications with root privileges, so it is quite simple for an attacker to get a root shell.
One of the most dangerous kind of control flow attacks consists of ROP (Return-Oriented Programming) techniques, or similar variants \cite{jop,no-ret}
which leverage different sequence of instructions equivalent to a return instruction.
Since code vulnerabilities may affect embedded systems \cite{schneider-bof,rockwell-vuln,rockwell-vuln2,elcsoft-vuln,siemens-s7}, ROP techniques
are applicable as well. Furthermore, due to the limitations imposed by these systems, is even more challenging to defend against them.


\section{Defenses}




\chapter{Pin Control Protection Design}
\label{chap:design}

TODO Pin Control Protection Design.


\chapter{Experimental Results}
\label{chap:results}

TODO Experimental Results.


\chapter{Conclusions}
\label{chap:conclusions}

This final chapter concludes the presented work, highlighting its main contribution and drawbacks, and suggesting possible future works.


\section{Contribution and drawbacks}
\label{sec:contrib}

The scientific contribution of this work is made of two main components:
\begin{enumerate}
	\item analyse I/O attack and prove that it is actually feasible on real PLCs;
	\item design and implement a possible countermeasure, achieving a good detection rate without causing an unacceptable performance overhead.
\end{enumerate}
Both parts can obviously be improved and continued, as discussed later in \mysec{sec:future}.

By analysing the attack, we demonstrated that it is possible to tamper with the I/O operations even in a real PLC architecture, where input/output is managed by external modules.
Furthermore, we were able to clearly define the requirements and the attack vectors available for the attacker, useful to abstract the problem and design a proper solution.
In the first part of \mychap{chap:defense}, we tried to define abstract strategies that could be applied to any embedded system to tackle any kind of I/O attack.
Furthermore, the implementation has been designed to minimise the effort required to adapt the solution to different systems and architectures.

Nevertheless, our detection system is still the first step against I/O attack, and, of course, it has some limitations.
First, it is not a complete solution, in the sense that I/O attacks could still be possible if the attacker is able to gain accurate timing precision and evade our time-based monitors.
Of course this raises the bar for the attacker, who has to conduct more elaborated attacks to achieve the same result.
Moreover, our detection monitors can be improved and optimized to increase their detection rate, factually discouraging any malicious attempt.
Second, an attacker who gains kernel level access may still be able to circumvent our defensive mechanism, and other techniques should be used in combination with it,
as previously discussed in \mysec{sec:def-sec}. In the next section we discuss some possible future works regarding both our contributions.


\section{Future works}
\label{sec:future}

During each phase of this work, many ideas about possible future works came out.
First, for the attack phase, further possibilities can be analysed to implement more elaborated attacks on the real PLC with external I/O modules.
For instance, in our Wago PLC, it may be possible to fake both input and output by leveraging SPI bitbanging technique.
A possible implementation of this attack can do the following:
\begin{itemize}
	\item synchronise itself with PLC I/O operation;
	\item disable next PLC I/O operation using one of the techniques presented in \mysec{sec:attack-plc};
	\item perform malicious I/O operation within the next scan cycle, \eg shifted by $\SI{5}{ms}$ after the real one (accurate synchronisation is required).
\end{itemize}
Note that all these operations can be done with the same assumptions made for the described I/O attack, without hooking any function nor modifying PLC logic/runtime.
Moreover, several vendors are currently producing different PLCs having a small subset of I/O interfaces directly available without the need of external modules.
If this feature, known as \emph{Integrated I/O}, is actively used on a control system, then the attacker can have direct access to sensors and actuators as well,
without needing to go through the additional level of indirection caused by external modules. In this case, the original version of Pin Control Attack would be applicable.

For the defense phase, many improvements and extension are doable.
First, detection rates of DR and I/O monitor can be improved by synchronising them with respect to the timing of the PLC scan cycle.
In fact, if they are triggered in proximity of each PLC I/O operation, detecting more sophisticated attacks becomes easier.
To get accurate results, this approach needs to be tested in a real-time system.
Of course, if the vendor decides to integrate the defense with the PLC runtime itself, this would be the optimal solution, because it can simply check I/O configuration
right before performing an I/O operation.

Another benefit deriving from an integration with the PLC runtime would be a significant simplification of our defense,
in particular when it has to check whether the I/O configuration is in line with the current PLC logic.
We largely discussed this aspect in \mysec{sec:io-design}. Briefly, if the PLC runtime is designed to be aware of the defense,
it can provide a simpler interface for Ghostbuster to check if a conflict between configuration and logic occurred.
In fact, in our prototype implementation, described in \mysec{sec:io-impl}, we leveraged reverse engineering techniques to collect the necessary information.

Additionally, our solution may be extended with a performance monitor, to cover a larger set of attacks and increase the overall detection rate.
We can argue that a performance-based mechanism would be effective on a PLC system, which executes the same operations at each scan cycle, thus, it is very stable.
As previously discussed in \mysec{sec:pre-analysis}, the operations performed by the PLC runtime when a new logic is uploaded should be excluded from the detection.
Again, an integration with the PLC runtime would be helpful to design this behaviour as well.
For instance, the performance monitor may be disabled and re-enabled by the PLC runtime process through authenticated commands sent to the defense module.
All these proposed approaches should be evaluated with respect to the overhead imposed to the system.

Finally, our entire solution can be deployed within a Trusted Execution Environment, such as ARM TrustZone \cite{trustzone}.
However, this approach may not be feasible due to the excessive CPU performance degradation to switch between secure and non-secure world.
Therefore, a carefully designed solution is required to minimise its overhead.



\begin{thebibliography}{99}

\bibitem{ghostplc}
A.Abbasi, M.Hashemi, E.Zambon, S.Etalle,
``Stealth Low-Level Manipulation of Programmable Logic Controllers I/O By Pin Control Exploitation'',
TODO Black Hat Conference 2016.



\end{thebibliography}



%\begin{appendices}
%\setcounter{table}{0}
%\setcounter{figure}{0}
%\renewcommand\thetable{\thesection\arabic{table}}
%\renewcommand\thefigure{\thesection\arabic{figure}}

%\end{appendices}


\end{document}
